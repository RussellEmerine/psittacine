Metaphors pervade human speech.
Many, many things are referred to by completely different things.
Parrots have metaphor too, and notably the most common metaphors
differ significantly from common human metaphors.

\subsection{Experiences as Flights}\label{subsec:experiences-as-flights}

Parrots are flying creatures.
While they are adept climbers without their wings,
they rely on flight for almost all their movement,
as much as humans rely on walking.
This leads to pervasive metaphor involving flight.

\begin{exe}
    \ex
    \glll
    rõk howeg=nosa \\
    mountain wind=\AblThree \\
    mountain wind=from \\
    \glt
    Wind comes from the mountain.
\end{exe}

This is like the English metaphor of ``the path ahead is long and difficult'',
but parrots don't use paths, they fly.
A headwind will make flight slower and more difficult,
while a tailwind will make flight faster and easier.

This particular sentence could be interpreted literally,
or idiomatically as ``Mountain-climbing is difficult'',
or perhaps as something else compared to mountain-climbing is difficult.

\subsection{Speech as Song}\label{subsec:speech-as-song}

Parrots' logical thinking does not occur in the cerebral cortex,
but rather in the HVC (an acronym that no longer has meaning),
which is the area of the brain responsible for birdsong.
Language will also be processed by the same brain region,
so parrots will consider language and song to be almost the same.
This results in some interesting lexical terms.
For example,

\begin{exe}
    \ex
    \glt
    sōkrohły lâl.
    \glll
    sōkro-hły lâl \\
    leaf-\Adj{} song \\
    leaf song \\
    \glt
    book
\end{exe}
A book is referred to as ``a song made of leaves'',
meaning that speech (song) is written onto pages (leaves).
