Verbs in Psittacine are rather simple morphologically.
They have some inflectional morphology but
do not change for agreement in person or number with the subject or object.
However, they do have more complicated syntactic mechanisms.

\subsection{Word Order}\label{subsec:word-order}

The typical word order is
\begin{center}
    \quad Verb \quad Subject \quad [\quad Direct Object \quad  [ \quad Indirect Object \quad ] \quad ]
\end{center}
where each verb takes a specific number of arguments.
The language is strictly nominative-accusative.
Word order is the only indication of which noun is the subject, direct object, or indirect object.

\begin{exe}
    \ex
    \glt
    kwǒn wihǔ.
    \glll
    kwǒn wihǔ \\
    sleep bird \\
    sleep bird \\
    \glt
    The bird sleeps.
\end{exe}

\begin{exe}
    \ex
    \glt
    krân wùg wug.
    \glll
    krân wùg wug \\
    see 1\Sg{} dog \\
    see me dog \\
    \glt
    I see the dog.
\end{exe}

\begin{exe}
    \ex
    \glt
    krân wug wùg.
    \glll
    krân wug wùg \\
    see dog 1\Sg{} \\
    see dog me \\
    \glt
    The dog sees me.
\end{exe}

\begin{exe}
    \ex
    \glt
    gāw gó hlātu wùg.
    \glll
    gāw gó hlātu wùg \\
    give 3\Sg{}.\Anim{} plate 1\Sg{} \\
    give him plate me \\
    \glt
    He gives me a plate.
\end{exe}

\subsection{Auxiliary Verbs}\label{subsec:auxiliary-verbs}

Verbs have an intrinsic valency.
Some are intransitive, some are transitive, and a few are ditransitive.
The language uses auxiliary verbs with gerunds to change transitivity.

Gerunds are formed by adding the suffix \textit{-ga} to a verb.

The auxiliary verb \textit{zà} ``do''
can take a transitive or ditransitive gerund as an object
and produce an intransitive form.
\textit{zà} is used exclusively for changing valency
and cannot be used for sentences like ``I do the homework''.

\begin{exe}
    \ex
    \glt
    zà wihǔ kōgga.
    \glll
    zà wihǔ kōg-ga \\
    do bird eat-\Ger{} \\
    do bird eating \\
    \glt
    The bird eats.
\end{exe}

\begin{exe}
    \ex
    \glt
    zà kwałŷx gāwga.
    \glll
    zà kwałŷx gāw-ga \\
    do tree give-\Ger{} \\
    do tree giving \\
    \glt
    The tree gives.
    /
    The tree provides.
\end{exe}

The auxiliary verb \textit{hłĩk} ``make, cause''
can take a gerund as an object.
If the gerund is possessed by a noun,
the noun is the patient of the sentence.
The causative formed is the most general type of causative,
where the patient is not necessarily forced.

In the following two examples,
\textit{xrîz} is naturally intransitive.

\begin{exe}
    \ex
    \glt
    hłĩk howeg kwałŷx xrîzgaty.
    \glll
    hłĩk howeg kwałŷx xrîz-ga=ty \\
    make wind tree shake-\Ger{}=\Poss{} \\
    make wind tree shaking=\Poss{} \\
    \glt
    The wind makes the tree shake.
    /
    The wind shakes the tree.
    \ex
    \glt
    hłĩk howeg xrîzga.
    \glll
    hłĩk howeg xrîz-ga \\
    make wind shake-\Ger{} \\
    make wind shaking \\
    \glt
    The wind shakes [things].
    \ex
    \glt
    hłîk wihǔ xǎw wihǔ kōggaty.
    \glll
    hłîk wihǔ xǎw wihǔ kōg-ga=ty \\
    make bird small bird eat-\Ger{}=\Poss{} \\
    make bird small bird eating=\Poss{} \\
    \glt
    The bird makes the chick eat.
    /
    The bird feeds the chick.
\end{exe}

The auxiliary verb \textit{xal} ``make, cause'',
similarly to \textit{hłĩk},
takes a gerund as an object.
However, \textit{xal} also takes in indirect object,
which acts as the direct object of the original transitive verb.

\begin{exe}
    \ex
    \glt
    xal wihǔ xǎw wihǔ kōggaty gò.
    \glll
    xal wihǔ xǎw wihǔ kōg-ga=ty gò \\
    make bird small bird eat-\Ger{}=\Poss{} fruit \\
    make bird small bird eating=\Poss{} fruit \\
    \glt
    The bird makes the chick eat fruit.
    /
    The bird feeds the chick fruit.
\end{exe}

\subsection{Replacements for be, have, and go}\label{subsec:replacements-for-be-have-and-go}

The language uses the zero copula.
This may be used to relate two nouns
or state a noun in a stance form.
Because of the language's complex directional system,
this takes the place of ``to be'', ``to go'', and ``to have''.

\begin{exe}
    \ex
    \glt
    wùg hełǎ.
    \glll
    wùg hełǎ \\
    1\Sg{} human \\
    me human \\
    \glt
    I am a human.
\end{exe}

\begin{exe}
    \ex
    \glt
    zine wùg zyla nũrty.
    \glll
    zine wùg zyla nũr=ty \\
    3\Sg{}.\Inanim{} 1\Sg{} blue feather=\Poss{} \\
    it me blue feather=\Poss{} \\
    \glt
    It's my blue feather.
\end{exe}

\begin{exe}
    \ex
    \glt
    wùg zyla nũrty.
    \glll
    wùg zyla nũr=ty \\
    1\Sg{} blue feather=\Poss{} \\
    me blue feather=\Poss{} \\
    \glt
    The blue feather is mine.
    /
    I have a blue feather.
\end{exe}

\begin{exe}
    \ex
    \glt
    crizǐ xârty słygizna.
    \glll
    crizǐ xâr=ty słygiz=na \\
    house top=\Poss{} vine=\AdessTwo{} \\
    house top=\Poss{} vine=towards \\
    \glt
    The vine goes up the house.
\end{exe}

Statement of existence equivalent to ``there is''
does not have a unique construction.
Rather, it is treated as a place having things.
If the statement of existence is too general
to be tied to a particular place,
``here'' is used by default,
or if that is ambiguous,
``the world'' is used.

\begin{exe}
    \ex
    \glt
    kwałŷx gòty.
    \glll
    kwałŷx gò=ty \\
    tree fruit=\Poss{} \\
    tree fruit=\Poss{} \\
    \glt
    The tree has fruit.
    /
    There is fruit in the tree.
\end{exe}

\begin{exe}
    \ex
    \glt
    kûk crizǐty.
    \glll
    kûk crizǐ=ty \\
    here house=\Poss{} \\
    here house=\Poss{} \\
    \glt
    There are houses here.
    /
    There are houses [in general].
\end{exe}

Adjectives act like intransitive verbs.

\begin{exe}
    \ex
    \glt
    zyla nũr.
    \glll
    zyla nũr \\
    blue feather \\
    blue feather \\
    \glt
    The feather is blue.
\end{exe}

\subsection{Negation}\label{subsec:negation}
A verb is negated by adding the suffix \textit{-hy}.

\begin{exe}
    \ex
    \glt
    lȳthy èt húgni.
    \glll
    lȳt-hy èt húg=ni \\
    fly-\Neg{} night bees=\InessTwo{} \\
    fly-not night bees=at \\
    \glt
    The bees do not fly at night.
\end{exe}

\begin{exe}
    \ex
    \glt
    kōghy wùg rãw.
    \glll
    kōg-hy wùg rãw \\
    eat-\Neg{} 1\Sg{} meat \\
    eat-not me meat \\
    \glt
    I do not eat meat.
\end{exe}

Since adjectives are like verbs, the suffix can be added directly to an adjective.

\begin{exe}
    \ex
    \glt
    zine twazwahłyhy.
    \glll
    zine twazwa-hły-hy \\
    3\Sg{}.\Inanim{} metal-\Adj{}-\Neg{} \\
    it metal-not \\
    \glt
    It's not made of metal.
\end{exe}

This also applies to adjectives that are not directly acting as verbs.
The most obvious application of this is in disambiguation,
but it can also just be a simple description (not quite an antonym, just a lack of a trait).
\begin{exe}
    \ex
    \glt
    kōghy wihǔ zighy gò.
    \glll
    kōg-hy wihǔ zig-hy gò \\
    eat-\Neg{} bird fresh-\Neg{} fruit \\
    eat-not bird fresh-not fruit \\
    \glt
    The birds do not eat unfresh fruit.
\end{exe}

Since there is no verb when relating two nouns
or stating a noun in a stance form,
a special verb must be used for negating such a statement.
\textit{ì} ``be / go'' is used
with the negation suffix attached.
\textit{ì} is special in that
it is neither transitive nor intransitive,
and accepts either two nouns or one noun in a stance form.

\begin{exe}
    \ex
    \glt
    ìhy wùg wihǔ.
    \glll
    ì-hy wùg wihǔ \\
    be-\Neg{} 1\Sg{} bird \\
    be-not me bird \\
    \glt
    I am not a bird.
\end{exe}

\begin{exe}
    \ex
    \glt
    ìhy èt kûkni twazwahły hūtna.
    \glll
    ì-hy èt kûk=ni twazwa-hły hūt=na \\
    be-\Neg{} night here=\InessTwo{} metal-\Adj{} caterpillar=\AdessTwo{} \\
    be-not night here=at metal caterpillar=towards \\
    \glt
    The trains do not go here at night.
\end{exe}

\subsection{Tense}\label{subsec:tense}
There are four tenses, present, past, future, and remote past,
used for example when telling stories.
The present is unmarked.
The other tenses are indicated by verbal suffixes.

\begin{exe}
    \ex
    \glt
    kwǒnce wùg.
    \glll
    kwǒn-ce wùg \\
    sleep-\Pst{} 1\Sg{} \\
    slept me \\
    \glt
    I slept.
\end{exe}

\begin{exe}
    \ex
    \glt
    zàcat takís gréga.
    \glll
    zà-cat takís gré-ga \\
    do-\Hst{} country farm-\Ger{} \\
    did country farming \\
    \glt
    The country farmed [once upon a time].
\end{exe}



The tense suffixes occur before \textit{-hy}.
\begin{exe}
    \ex
    \glt
    xwèwykhy wùg agìty zine.
    \glll
    xwè-wyk-hy wùg agì=ty zine \\
    know-\Fut{}-\Neg{} 1\Sg{} friend=\Poss{} 3\Sg{}.\Inanim{} \\
    know-will-not me friend=\Poss{} it \\
    \glt
    My friend will not know it.
\end{exe}

``be'', ``have'', and ``go''
are usually expressed without a verb in the present tense.
In the other tenses, they must have a verb.
Forms which take one noun in a stance form use the same \textit{ì}
that negation uses.
However, forms which relate two nouns
use the suppletive verb \textit{sèt}, which is not used in present tense.

\begin{exe}
    \ex
    \glt
    sètwyk hełǎ wùg agìty.
    \glll
    sèt-wyk hełǎ wùg agì=ty \\
    be-\Fut{} human 1\Sg{} friend=\Poss{} \\
    be-will human me friend=\Poss{} \\
    \glt
    The human will be my friend.
\end{exe}

\begin{exe}
    \ex
    \glt
    ìce lutlùt wùgna.
    \glll
    ì-ce lutlùt wùg=na \\
    be-\Pst{} river 1\Sg{}=\AdessTwo{} \\
    was river me=at \\
    \glt
    I was at the river.
\end{exe}

\begin{exe}
    \ex
    \glt
    ìce lutlùt wùgza.
    \glll
    ì-ce lutlùt wùg=za \\
    be-\Pst{} river 1\Sg{}=\AllTwo{} \\
    went river me=to \\
    \glt
    I went to the river.
\end{exe}

\begin{exe}
    \ex
    \glt
    ìcat à gónty.
    \glll
    ì-cat à gón=ty \\
    be-\Fut{} 2\Sg{} food=\Poss{} \\
    have-will you food=\Poss{} \\
    \glt
    The food will be yours.
    /
    You will have food.
\end{exe}

Adjectives simply take tense suffixes like normal verbs.

\begin{exe}
    \ex
    \glt
    xǎwcat wùgwùg.
    \glll
    xǎw-cat wùgwùg \\
    young-\Hst{} 1\Pl{} \\
    young-were we \\
    \glt
    We were young [long ago, or in a story].
\end{exe}

\subsection{Aspect and Mood}\label{subsec:aspect-and-mood}
Semantic aspect and mood are not indicated grammatically.
Rather, if they have reason to be expressed, they are just adverbs
(or sometimes complex constructions if more detail is required).

Example for progressive aspect.
\begin{exe}
    \ex
    \glt
    taktak lȳtce wihǔ.
    \glll
    taktak lȳt-ce wihǔ \\
    for.some.time fly-\Pst{} bird \\
    for.some.time flew bird \\
    \glt
    The bird flew for some time.
    /
    The bird was flying.
\end{exe}

Example for iterative aspect.
\begin{exe}
    \ex
    \glt
    ò xenãtce wùg sōkrohły lâl.
    \glll
    ò xenãt-ce wùg sōkro-hły lâl \\
    again read-\Pst{} 1\Sg{} leaf-\Adj{} song \\
    again read me leafy song \\
    \glt
    I read the book again.
    /
    I reread the book.
\end{exe}

Example for potential mood.
Most adverbs don't apply to the copula in a natural way,
but this is an instance where it can happen.
Adverbs applied to the null copula just end up at the start of the sentence.
\begin{exe}
    \ex
    \glt
    cēłta zine rõk.
    \glll
    cēłta zine rõk \\
    might 3\Sg{}.\Inanim{} mountain \\
    might it mountain \\
    \glt
    It might be the mountain.
\end{exe}

