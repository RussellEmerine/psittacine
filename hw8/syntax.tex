Basic word order has already been explained.
Once again, it is
\begin{center}
    \quad Verb \quad Subject \quad [\quad Direct Object \quad  [ \quad Indirect Object \quad ] \quad ]
\end{center}
There are a number of other syntactic mechanisms for more structurally complex sentences.

\subsection{Conjunctions and Conditionals}\label{subsec:conjunctions-and-conditionals}

The basic conjunctions are \textit{tot} ``and''
and \textit{cec} ``or''.
When joining two items, the conjunction is placed between.
When joining three or more items,
the conjunction may be placed between each item
or may be used just once after all the items.
Parallel items all undergo any expected inflection.

\begin{exe}
    \ex
    \glt
    cǎwce nī tot wihǔ.
    \glll
    cǎw-ce nī tot wihǔ \\
    loud-\Pst{} cat and bird \\
    loud-were cat and bird \\
    \glt
    The cat and the bird were loud.
\end{exe}

(\textit{kȳhi} is an adverb.)
\begin{exe}
    \ex
    \glt
    kȳhi à húgty hūtty crákty cec.
    \glll
    kȳhi à húg=ty hūt=ty crák=ty cec \\
    able 2\Sg{} bee=\Poss{} caterpillar=\Poss{} ant=\Poss{} or \\
    able you bee=\Poss{} caterpillar=\Poss{} ant=\Poss{} or \\
    \glt
    You can have the bee, the caterpillar, or the ant.
\end{exe}

\begin{exe}
    \ex
    \glt
    lȳtce, toktõkce, slêxce tot wùgwùg.
    \glll
    lȳt-ce toktõk-ce slêx-ce tot wùgwùg \\
    fly-\Pst{} run-\Pst{} swim-\Pst{} and 1\Pl{} \\
    flew ran swam and us \\
    \glt
    We flew, ran, and swam.
\end{exe}

This is also how clauses are joined by conjunctions.

\begin{exe}
    \ex
    \glt
    toktõkce nī, tot gēntyce wug nī.
    \glll
    toktõk-ce nī tot gēnty-ce wug nī \\
    run-\Pst{} cat and follow-\Pst{} dog cat \\
    ran cat and followed dog cat \\
    \glt
    The cat ran, and the dog followed the cat.
\end{exe}

\begin{exe}
    \ex
    \glt
    krânce wihǔ nī, krânce nī wug, krânce wug hełǎ tot \\
    \glll
    krân-ce wihǔ nī krân-ce nī wug krân-ce wug hełǎ tot \\
    see-\Pst{} bird cat see-\Pst{} cat dog see-\Pst{} dog human and \\
    saw bird cat saw cat dog saw dog human and \\
    \glt
    The bird saw the cat, the cat saw the dog, and the dog saw the human.
\end{exe}

The emphasized forms ``either \dots or''
and ``both \dots and''
can be expressed by placing the conjunction between the words and after the list.

\begin{exe}
    \ex
    \glt
    nácce wùg sōkrohły lâl tot zū sōkro tot.
    \glll
    nác-ce wùg sōkro-hły lâl tot zū sōkro tot \\
    take-\Pst{} 1\Sg{} leaf-\Adj{} song and thick leaf and \\
    took me leafy song and thick leaf and \\
    \glt
    I took both the book and the card.
\end{exe}

Conditional compound sentences are formed
similarly to conjunctive compound sentences,
by putting the clauses on either side of a linking word.
In English, the words relating the clauses can occur in various places,
e.g.\ ``If X, then Y'' vs.\ ``When X, Y'' vs.\ ``X, yet Y''.
I instead have just one word that is always between the two sides.

English has special rules for how tenses are expressed under irrealis moods
(specifically, ``if I were'' is the prescribed standard).
I choose for tenses to be expressed based only on time.

\begin{exe}
    \ex
    \glt
    xal à łỹz tāgagaty, xīg kùwixwyk wùg zinezine.
    \glll
    xal à łỹz tāga-ga=ty xīg kùwix-wyk wùg zinezine \\
    make 2\Sg{} flower grow-\Ger{}=\Poss{} if.then buy-\Fut{} 1\Sg{} 3\Pl{}.\Inanim{} \\
    make you flower growing=\Poss{} if.then buy-will me them \\
    \glt
    If you grow flowers, I will buy them.
\end{exe}

\begin{exe}
    \ex
    \glt
    nácce gó rõrhły cíntag, wên gǔzce zine lúkty rygi.
    \glll
    nác-ce gó rõr-hły cíntag wên gǔz-ce zine lúk=ty ryg=i \\
    take-\Pst{} 3\Sg{}.\Anim{} gold-\Adj{} statue when.then put-\Pst{} 3\Sg{}.\Inanim{} place=\Poss{} sand=\IllThree{} \\
    took him golden statue when.then put it place=\Poss{} sand=\IllThree{} \\
    \glt
    When he took the golden statue, he put sand in its place.
\end{exe}

\subsection{Subordinate Clauses}\label{subsec:subordinate-clauses}

A subordinate clause describing a noun (i.e.\ a relative clause)
is formed by using the clause in its standard form,
replacing each referent to the noun with
the appropriate demonstrative pronoun form of ``that'',
and then following the clause with
the determiner form of ``that'' and the noun.
If the demonstrative pronoun comes right before
the demonstrative determiner, the pronoun can be dropped.

\begin{exe}
    \ex
    \glt
    krân wùg xłǒsce à krũxhły rõk.
    \glll
    krân wùg xłǒs-ce à krũx-hły rõk \\
    see me draw-\Pst{} 2\Sg{} \Dem{}.\Dist{}.\Inanim{}-\Adj{} mountain \\
    see me drew you that mountain \\
    \glt
    I see the mountain that you drew.
\end{exe}

\begin{exe}
    \ex
    \glt
    zine zīghy krũx kȳt krũxhły kiłìl.
    \glll
    zine zīg-hy krũx kȳt krũx-hły kiłìl \\
    3\Sg{}.\Inanim{} harm-\Neg{} \Dem.\Dist{}.\Inanim{}.\Sg{} 4\Sg{} \Dem.\Dist{}.\Inanim{}-\Adj{} secret \\
    it harm-not that one secret \\
    \glt
    It is a secret that does not harm one.
\end{exe}

(I translate \textit{sît} as ``seek'' in the gloss since it is transitive,
but as ``search'' in the translation since that is the translation
that is more faithful to meaning.)
\begin{exe}
    \ex
    \glt
    câwtuce zàce kũx sîtga kũxhły húg łỹz.
    \glll
    câwtu-ce zà-ce kũx sît-ga kũx-hły húg łỹz \\
    find-\Pst{} do-\Pst{} \Dem{}.\Dist{}.\Anim{}.\Sg{} seek-\Ger{} \Dem{}.\Dist{}.\Anim{}-\Adj{} bee flower \\
    found did that seeking that bee flower \\
    \glt
    The bee that searched found the flower.
\end{exe}

Instrumentals are expressed using this form.
To say ``A did B with C'',
use ``A that use[d] C did B'' (in the approprate tense).

\begin{exe}
    \ex
    \glt
    tèkce krũx kwal õgce krũxhły lutlùt rõk.
    \glll
    tèk-ce krũx kwal õg-ce krũx-hły lutlùt rõk \\
    hit-\Pst{} \Dem{}.\Dist{}.\Inanim{}.\Sg{} use-\Pst{} water \Dem{}.\Dist{}.\Inanim{}-\Adj{} river mountain \\
    hit that used water that river mountain \\
    \glt
    The river hit the mountain with water.
\end{exe}

A subordinate clause acting as a noun
is expressed the same way,but just with the demonstrative pronoun
at the end, rather than the demonstrative determiner and a noun.

\begin{exe}
    \ex
    câwtuce wùg sîtce wùg krũx.
    \glll
    câwtu-ce wùg sît-ce wùg krũx \\
    find-\Pst{} 1\Sg{} seek-\Pst{} 1\Sg{} \Dem{}.\Inanim{}.\Dist{}.\Sg{} \\
    found me sought me that  \\
    \glt
    I found what I was searching for.
\end{exe}

\begin{exe}
    \ex
    nôrhły él hēwakce kũx kwal kũx.
    \glll
    nôr-hły él hēwak-ce kũx kwal kũx \\
    \Dem{}.\Anim{}.\Prox{}-\Adj{} person drink-\Pst{} \Dem{}.\Anim{}.\Dist{}.\Sg{} water \Dem{}.\Anim{}.\Dist{}.\Sg{} \\
    this person drank that water that \\
    \glt
    This person is who drank the water.
\end{exe}

A clause acting as a noun (i.e.\ a content clause)
is expressed by simply placing the subordinate clause directly within the outer clause.
This is similar to the English form that elides ``that''.

\begin{exe}
    \ex
    \glt
    xwè wùg krânce gó wùgwùg.
    \glll
    xwè wùg krân-ce gó wùgwùg \\
    know 1\Sg{} see-\Pst{} 3\Sg{}.\Anim{} 1\Pl{} \\
    know me saw him us \\
    \glt
    I know that he saw us.
    /
    I know he saw us.
\end{exe}

\subsection{Questions}\label{subsec:questions}
All questions have the question particle \textit{ã} at the front.

A polar question is formed by adding the question particle \textit{ã}
at the front of the sentence,
and adding \textit{ha} ``yes'' or \textit{ìk} ``no'' to the end.
There is no significant difference between the two options
(meaning neither is the expected answer).

\begin{exe}
    \ex
    \glt
    ã, câwtuce gógó wug, ha?
    \glll
    ã câwtu-ce gógó wug ha \\
    \Q{} find-\Pst{} 3\Pl{}.\Anim{} dog yes \\
    \Q{} found them dog yes \\
    \glt
    Did they find the dog?
\end{exe}

\begin{exe}
    \ex
    \glt
    ã, kōgce gréryl kǒg?
    \glll
    ã kōg-ce gré-ryl kǒg \\
    \Q{} eat-\Pst{} farm-\Agt{}.\Anim{} grain \\
    \Q{} ate farmer grain \\
    \glt
    Did the farmer eat the grain?
\end{exe}

An open question is formed by adding the question particle \textit{ã}
at the front of the sentence,
and using the interrogative pronoun \textit{twãx} for inanimate topics
and \textit{tãx} for animate topics.
Just like other pronouns, these reduplicate for plurals
and take the \textit{-hły} adjectival suffix to form determiners.
(\Int{} means interrogative pronoun.)

\begin{exe}
    \ex
    \glt
    ã, câwtuce twãx gógóni wug?
    \glll
    ã câwtu-ce twãx gógó=ni wug \\
    \Q{} find-\Pst{} \Int{}.\Inanim{}.\Sg{} 3\Pl{}.\Anim{}=\InessTwo{} dog \\
    \Q{} found what them=at dog \\
    \glt
    Where did they find the dog?
\end{exe}

\begin{exe}
    \ex
    \glt
    ã, hłîkce tãxtãx à krângaty sōkro?
    \glll
    ã hłîk-ce tãxtãx à krân-ga=ty sōkro \\
    \Q{} make-\Pst{} \Int{}.\Anim{}.\Pl{} 2\Sg{} see-\Ger{}=\Poss{} leaf \\
    \Q{} made who you seeing=\Poss{} leaf \\
    \glt
    Who (pl.) showed you the leaf?
\end{exe}

\begin{exe}
    \ex
    \glt
    ã, kōgce à twãx?
    \glll
    ã kōg-ce à twãx \\
    \Q{} eat-\Pst{} 2\Sg{} \Int{}.\Inanim{}.\Sg{} \\
    \Q{} ate you what \\
    \glt
    What did you eat?
\end{exe}

\begin{exe}
    \ex
    \glt
    ã, lāl à twãx-hły lâl?
    \glll
    ã lāl à twãx-hły lâl \\
    \Q{} sing 2\Sg{} \Int{}.\Inanim{}-\Adj{} song \\
    \Q{} sing you what song \\
    \glt
    What song are you singing?
\end{exe}

For questions with options,
simply list the options at the end of the question,
joining them with ``or''.

\begin{exe}
    \ex
    ã, kazcǒ à twãxhły, hlātu, cycīl, zyla cec?
    \glll
    ã kazcǒ à twãx-hły hlātu cycīl zyla cec \\
    \Q{} want 2\Sg{} \Int{}.\Inanim{}-\Adj{} plate red blue white or \\
    \Q{} want you what plate red blue white or \\
    \glt
    Which plate do you want, red, blue, or white?
\end{exe}

