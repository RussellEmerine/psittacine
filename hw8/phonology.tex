\subsection{Consonants}\label{subsec:consonants}

Parrots do not have the same mouths as humans.
Also, parrots do not have a voice-producing larynx like humans do.
They instead produce sound at the syrinx,
an organ at the fork of the trachea found in birds.
While syrinxes are more powerful and flexible than larynxes,
the sounds they make are similar to human speech sounds for the purposes of notation.
So, sounds will be written as if they used standard human place and manner of articulation.

The biggest articulatory difference between humans and parrots is that
parrots don't have lips, so they are not able to produce labials.
In videos I found, experienced parrots can make labial-sounding sounds,
but they aren't created using the edge of the beak,
but by some other mechanism, perhaps the tongue.
Psittacine will not have labials at all.

Some other notable features I noticed in videos are that
parrots have very pronounced \textipa{[\:R]} and \textipa{[l]} sounds,
and don't make human-like trills.
I chose to expand on the sound of strong approximants by including more approximants.
I chose for there to only be voiceless plosives and fricatives,
to produce a more ``chitter-chatter'' kind of sound.

\begin{center}
    \begin{tabular}{|c|c|c|c|c|}
        \hline
        & Alveolar
        & Retroflex
        & Velar
        & Glottal \\
        \hline
        Plosive
        & \textipa{/t/} t
        &
        & \textipa{/k/} k
        & \\
        \hline
        Nasal
        & \textipa{/n/} n
        &
        & \textipa{/N/} g
        & \\
        \hline
        Fricative
        & \textipa{/s/} s
        & \textipa{/\:s/} x
        &
        & \textipa{/h/} h \\
        \hline
        Affricate
        & \textipa{/\t{ts}/} z
        & \textipa{/\t{t\:s}/} c
        &
        & \\
        \hline
        Approximant
        &
        & \textipa{/\:R/} r
        & \textipa{/w/} w
        & \\
        \hline
        Lateral Approximant
        & \textipa{/l/} l
        &
        & \textipa{/\L/} ł
        & \\
        \hline
    \end{tabular}
\end{center}

\subsection{Vowels}\label{subsec:vowels}

Parrots have a complete range of vowel qualities,
with no significant difference from humans.
While they don't have lips, they can make rounded vowels with no problems,
possibly using their tongue.
I chose the following inventory because just because it contains both
\textipa{/\ae/} and \textipa{/A/},
which I plan to use in specific vocabulary items,
and it isn't too imbalanced.

\begin{center}
    \begin{tabular}{|c|c|c|c|}
        \hline
        & Front             & Mid             & Back            \\
        \hline
        High & \textipa{/i/} i   & \textipa{/1/} y & \textipa{/u/} u \\
        \hline
        Mid  & \textipa{/e/} e   &                 &                 \\
        \hline
        Low  & \textipa{/\ae/} a &                 & \textipa{/A/} o \\
        \hline
    \end{tabular}
\end{center}

There are no diphthongs.
There are some instances of adjacent vowels,
all pronounced in hiatus.

\subsection{Tones and Phonation}\label{subsec:tones-and-phonation}

Since parrot speech is similar to birdsong,
Psittacine will have a relatively high number of tones.
Parrots also have atypical phonation.
Creaky voice seems to be especially common in the videos I saw.
(Interestly, the combination of tone contours and phonation is also present in Vietnamese.)

Tones will only be applied to approximately one syllable per content morpheme,
in a manner similar to the toned pitch accent system in Norwegian and Swedish.
The tones on the accented syllables may allophonically affect tone in surrounding syllables,
i.e.\ a low tone start on an accented syllable may make the syllable before also low.

The following tones and phonations will be available on all vowels.
They are demonstrated on \textipa{/A/}.

\begin{center}
    \begin{tabular}{|c|c|}
        \hline
        Description   & Transcription               \\
        \hline
        Mid tone      & \textipa{/A\tone{22}/} o    \\
        \hline
        High tone     & \textipa{/A\tone{55}/} ō    \\
        \hline
        Rising tone   & \textipa{/A\tone{35}/} ó    \\
        \hline
        Falling tone  & \textipa{/A\tone{51}/} ò    \\
        \hline
        Peaking tone  & \textipa{/A\tone{352}/} ô   \\
        \hline
        Creaky low    & \textipa{/\~*A\tone{11}/} õ \\
        \hline
        Creaky rising & \textipa{/\~*A\tone{13}/} ǒ \\
        \hline
    \end{tabular}
\end{center}

\subsection{Phonotactics}\label{subsec:phonotactics}

The syllable structure of the language is (C)(R)V(C),
where C is a consonant, R is a (possibly lateral) approximant, and V is a vowel,
with some tone if it is accented.
When there is a syllable-initial consonant cluster,
only the following CR groups are possible, chosen based on ease of pronunciation:
\begin{center}
    \begin{tabular}{|c|c|c|c|c|c|c|c|c|c|}
        \hline
        & t          & k          & n          & g          & s          & x          & h          & z          & c          \\
        \hline
        r &            & \checkmark &            & \checkmark &            & \checkmark & \checkmark &            & \checkmark \\
        \hline
        w & \checkmark & \checkmark & \checkmark & \checkmark & \checkmark & \checkmark & \checkmark & \checkmark & \checkmark \\
        \hline
        l &            & \checkmark &            &            & \checkmark & \checkmark & \checkmark &            &            \\
        \hline
        ł &            & \checkmark &            &            & \checkmark & \checkmark & \checkmark &            &            \\
        \hline
    \end{tabular}
\end{center}

Otherwise, any consonant can start a syllable.

Any consonant other than h can end a syllable.

The central vowel is intended to be just one vowel, not a diphthong.

\subsection{Example Phonotactically Sound Words}\label{subsec:example-phonotactically-sound-words}
\begin{itemize}
    \item wùg \textipa{/wuN\tone{51}/} ``me''
    \item słygiz \textipa{/s\L{}1Ni\t{ts}\tone{33}/} ``vine''
    \item hengó \textipa{/henNA\tone{35}/} ``forest''
    \item rõk \textipa{/\:R\~*Ak\tone{11}/} ``mountain''
    \item kłas \textipa{/k\L{}\~*\ae{}s\tone{13}/} ``glass''
\end{itemize}