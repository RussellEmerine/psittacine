One instance of derivational morphology
has already been explained in a previous section,
which is \textit{-hły},
a suffix that forms adjectives out of materials.
It is overloaded to create determiner forms
of demonstrative pronouns.

\begin{exe}
    \ex
    ã, kazcǒ à twãxhły, hlātu, cycīl, zyla?
    \glll
    ã kazcǒ à twãx-hły hlātu cycīl zyla  \\
    \Q{} want 2\Sg{} \Int{}.\Inanim{}-\Adj{} plate red blue white \\
    build-\Hst{} bird grassy nest, wooden nest, stone nest and \\
    \glt
    The birds built a grass nest, a wooden nest, and a stone nest.
\end{exe}

\textit{-kac} takes an adjective and turns it into a verb
with ``become'', like the intransitive forms of
the English suffixes ``-ify'' and ``-ize''.
This is semantically somewhat similar to the auxiliary forms from earlier,
but I chose a different mechanism because it only applies to adjectives.

\begin{exe}
    \ex
    cycīlkacce zān.
    \glll
    cycīl-kac-ce zān \\
    red-become-\Pst{} sky \\
    became.red sky \\
    \glt
    The sky became red.
    /
    The sky reddened.
\end{exe}

\begin{exe}
    \ex
    kłàzkacce zine wên gūskacce zine.
    \glll
    kłàz-kac-ce zine wên gūs-kac-ce zine \\
    dry-become-\Pst{} 3\Sg{}.\Inanim{} when.then small-become-\Pst{} it \\
    dried it when.then shrank it \\
    \glt
    When it dried, it shrank.
\end{exe}

\textit{-ryl} takes a verb and makes an animate agentive noun.
\textit{-łyr} takes a verb and makes an inanimate agentive noun.

\begin{exe}
    \ex
    xwè slêxryl kwal.
    \glll
    xwè slêx-ryl kwal \\
    know swim-\Agt{}.\Anim{} water \\
    know swimmer water \\
    \glt
    The swimmer knows the water.
\end{exe}

\begin{exe}
    \ex
    sîtce kũx õgce tylǐnłyr kũxhły wùg rõr
    \glll
    sît-ce kũx õg-ce tylǐn-łyr kũx-hły wùg rõr \\
    seek-\Pst{} \Dem{}.\Anim{}.\Sg{} use-\Pst{} dig-\Agt{}.\Inanim{} \Dem{}.\Anim{}-\Adj{} 1\Sg{} gold \\
    sought that used spade that me gold \\
    \glt
    I searched for gold with the spade.
\end{exe}

\textit{-zoc} takes a verb or adjective and forms a noun
representing the process or result of the action (like ``-tion'' or the ``-th'' in ``growth'' and ``theft'')
or the state of the adjective (like ``-ness'').

\begin{exe}
    \ex
    kȳhihy kùwix kȳt cỳltyzoc.
    \glll
    kȳhi-hy kùwix kȳt cỳlty-zoc \\
    able-\Neg{} buy 4\Sg{} happy-\Nmlz{} \\
    able-not buy one happiness \\
    \glt
    One cannot buy happiness.
\end{exe}

\begin{exe}
    \ex
    ã, twãxtwãx à câwtuzocty?
    \glll
    ã twãxtwãx à câwtu-zoc=ty \\
    \Q{} \Int{}.\Inanim{}.\Pl{} 2\Sg{} find-\Nmlz{}=\Poss{} \\
    \Q{} what you findings=\Poss{} \\
    \glt
    What are your findings?
\end{exe}

\textit{-łon} takes a noun and forms an adjective of similarity,
like ``-like'' in English.

\begin{exe}
    \ex
    kiłìlłon wùgwùg xwẽ crizǐtyty.
    \glll
    kiłìl-łon wùgwùg xwẽ crizǐ=ty=ty \\
    secret-like 1\Pl{} study house=\Poss{}=\Poss{} \\
    secretive us study house=\Poss{}=\Poss{} \\
    \glt
    Our school is secretive.
\end{exe}

\begin{exe}
    \ex
    sû nīłon à wugty.
    \glll
    sû nīłon à wug=ty \\
    very cat-like you dog=\Poss{} \\
    very catlike you dog=\Poss{} \\
    \glt
    Your dog is very catlike.
\end{exe}

\textit{-et} takes a verb and forms an adjective that
usually means a specialized or generalized version of the past participle.
As in English, the participle applies to the object of a transitive verb
or the subject of an intransitive verb.
(Wikipedia calls this behavior some distinction between active and passive uses.)

\begin{exe}
    \ex
    ōt tùl tùlkacet sōkro kwalty.
    \glll
    ōt tùl tùl-kac-et sōkro kwal=ty \\
    should sweet sweet-become-\Ptcp{} leaf water=\Poss{} \\
    should sweet sweetened leaf water=\Poss{} \\
    \glt
    The sweetened tea should be sweet.
\end{exe}

\begin{exe}
    \ex
    ã, kōgwyk à gréet gò, ha?
    \glll
    ã kōg-wyk à gré-et gò ha \\
    \Q{} eat-\Fut{} 2\Sg{} farm-\Ptcp{} fruit yes \\
    \Q{} eat-will you farmed fruit yes \\
    \glt
    Will you eat the farmed fruit?
\end{exe}

\begin{exe}
    \ex
    xrôk tāgaet wihǔ.
    \glll
    xrôk tāga-et wihǔ \\
    big grow-\Ptcp{} bird \\
    big grown bird \\
    \glt
    The grown bird is big.
\end{exe}