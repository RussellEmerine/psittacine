Adjectives are very simple.
They are simply placed before the noun.

\subsection{Adjectivization}\label{subsec:adjectivization}
The -hły suffix marks adjectivization.
The most important use of -hły is be turning material nouns into adjectives.
As shown in the section about pronouns,
-hły is also used to turn demonstrative pronouns into demonstrative determiners,
which also act as normal adjectives.

\begin{exe}
    \ex
    \glt
    xrõk gón slôrhły hlātula
    \glll
    xrõk gón slôr-hły hlātu=la \\
    amazing food \Dem{}.\Prox{}-\Adj{} plate=\AdessThree{} \\
    amazing food this plate=on \\
    \glt
    the amazing food on this plate
\end{exe}
This example is ambiguous;
the above phrase also could mean
``the amazing food on the plate made of this [material]''.

\begin{exe}
    \ex
    \glt
    twazwahły wihǔ
    \glll
    twazwa-hły wihǔ \\
    metal-\Adj{} bird \\
    metal bird \\
    \glt
    airplane
    \ex
    \glt
    kłǎshły nūr
    \glll
    kłǎs-hły nūr \\
    glass-\Adj{} light \\
    glass light \\
    \glt
    lightbulb
\end{exe}

A number of terms that may be simple or compound words in English
are referred to with adjective-noun phrases in Psittacine.
Material adjectives are common for this.

\subsection{Adverbs}\label{subsec:adverbs}

Adverbs are just adjectives placed before verbs rather than before nouns.

\begin{exe}
    \ex
    \glt
    cỳłty kwǒn wùg.
    \glll
    cỳłty kwǒn wùg \\
    happy sleep 1\Sg{} \\
    happy sleep me \\
    \glt
    I sleep happily.
\end{exe}

