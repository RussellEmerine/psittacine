just storage for now


\subsection{Sample Sentences}\label{subsec:sample-sentences}

I have intentionally made some words with
some semantic relation similar but with different tones.
I explain this as historically old morphology
that was lost and only reflected in tones,
then completely obscured after tone changes over time.
In my conworld, parrots have only recently begun
speaking human-like languages.
I explain the very fast rates of change
as volatility within a very new language system.

\begin{exe}
    \ex
    kwīn kwĩn.
    \glll
    kwīn kwĩn \\
    sour pickle \\
    sour pickle \\
    \glt
    The pickle is sour.
\end{exe}

Since the language does not have zero-valency verbs
and I do not like English's use of dummy pronouns,
in cases like weather statements
I use a reasonable subject.

\begin{exe}
    \ex
    nùr tyłǎc gás ozkã cìz.
    \glll
    nùr tyłǎc gás ozkã cìz \\
    bright sun but cold air \\
    bright sun but cold air \\
    \glt
    The sun is bright, but it's cold.
\end{exe}

Double negation is still negative.
In this case, it is not required everywhere.
There are several possibilities that must be clarified.
\begin{itemize}
    \item ``already-not bloom'' means they are just starting to bloom
    (more idiomatic to use a ``just starting'' adverb).
    \item ``already bloom-not'' means they are no longer blooming.
    \item ``already-not bloom-not'' means have not yet bloomed.
    \item ``already-not bloomed'' means they were starting to bloom.
    \item ``already bloomed-not'' means they had stopped blooming.
    \item ``already-not bloomed-not'' means they had not yet bloomed.
\end{itemize}
\begin{exe}
    \ex
    ǐci łŷzce hwàgkin łỹz, gás ǐcihy łŷzhy zyla łỹz.
    \glll
    ǐci łŷz-ce hwàgkin łỹz gás ǐci-hy łŷz-hy zyla łỹz \\
    already bloom-\Pst{} yellow flower but already-\Neg{} bloom-\Neg{} blue flower \\
    already bloomed yellow flower but already-not bloom-not blue flower \\
    \glt
    The yellow flowers have bloomed, but the blue flowers have not.
\end{exe}

\begin{exe}
    \ex
    ǐci łŷzhy cycīl łỹz.
    \glll
    ǐci łŷz-hy cycīl łỹz \\
    already bloom-\Neg{} red flower \\
    already bloom-not red flower \\
    \glt
    The red flowers already are not blooming.
    /
    The red flowers have stopped blooming.
\end{exe}

To say it is some season, say the day is of the season.
\begin{exe}
    \ex
    howeg zasõgty kàtty tot cycīlkac sōkro.
    \glll
    howeg zasõg=ty kàt=ty tot cycīl-kac sōkro \\
    wind season=\Poss{} day=\Poss{} and red-become leaf \\
    wind season=\Poss{} day=\Poss{} and redden leaf \\
    \glt
    It is autumn and the leaves are turning red.
\end{exe}

\begin{exe}
    \ex
    kōg tyłǎc zasõgty wùgwùgni āx gò.
    \glll
    kōg tyłǎc zasõg=ty wùgwùg=ni āx gò \\
    eat sun season=\Poss{} 1\Pl{}=\InessTwo{} much fruit \\
    eat sun season=\Poss{} us=in much fruit \\
    \glt
    We eat a lot of fruit in summer.
\end{exe}

\begin{exe}
    \ex
    xãk gógó kwałŷxty xârtyi.
    \glll
    xãk gógó kwałŷx=ty xâr=ty=i \\
    bottom 3\Pl{}.\Anim{} tree=\Poss{} top=\Poss{}=\IllThree{} \\
    bottom them tree=\Poss{} top=\Poss{}=into \\
    \glt
    Their trees' tops go down.
\end{exe}

(The language does have separate words for big, many, and much.)
\begin{exe}
    \ex
    lỳ sōkro gás łỹz tôctyn sỳcty.
    \glll
    lỳ sōkro gás łỹz tôctyn sỳc=ty \\
    green leaf but flower many color=\Poss{} \\
    green leaf but flower many color=\Poss{} \\
    \glt
    Leaves are green but flowers have many colors.
\end{exe}

\begin{exe}
    \ex
    ǐxa łŷz tot slyc słarýhły kwałŷxhły hengó.
    \glll
    ǐxa łŷz tot slyc słarý-hły kwałŷx-hły hengó \\
    sudden bloom and die grass-\Adj{} tree-\Adj{} forest \\
    sudden bloom and die grass tree forest \\
    \glt
    A bamboo forest blooms and dies suddenly.
\end{exe}

(
Wind instruments are called ``[inanimate] singers''
since, compared to other instruments,
their way of making sound is similar to that of humans and parrots.
)
\begin{exe}
    \ex
    hłīk cìz lālłyr słarýhły kwałŷxhły tàek xrîzgaty.
    \glll
    hłīk cìz lāl-łyr słarý-hły kwałŷx-hły tàek xrîz-ga=ty \\
    make air sing-\Agt{}.\Inanim{} grass-\Adj{} tree-\Adj{} chip shake-\Ger{}=\Poss{} \\
    make air singer grass tree chip shaking=\Poss{} \\
    \glt
    Air shakes a wind instrument's reed.
\end{exe}

\begin{exe}
    \ex
    wugłon õgce wùg krũxhły kláte.
    \glll
    wug-łon õg-ce wùg krũx-hły kláte \\
    dog-like use-\Pst{} 1\Sg{} \Dem{}.\Inanim{}.\Dist{}-\Adj{} design \\
    dog-like used me that design \\
    \glt
    The design I used is dog-like.
\end{exe}

\begin{exe}
    \ex
    kazcǒhy hàgry kȳtkȳt xār xwẽty.
    \glll
    kazcǒ-hy hàgry kȳtkȳt xār xwẽ=ty \\
    want-not some 4\Pl{} brain study=\Poss{} \\
    want-not some them brain study=\Poss{} \\
    \glt
    Some don't want to use psychology.
\end{exe}

\begin{exe}
    \ex
    kazcǒhy hàgry kȳtkȳt xār xwẽty.
    \glll
    xrôł kȳtkȳt kùwix él krârwik \\
    say 4\Pl{} buy person idea \\
    say them buy person idea \\
    \glt
    They say people buy ideas.
\end{exe}

\begin{exe}
    \ex
    cỳlty cǎw nùr tot lâl gás kwãlhy zine.
    \glll
    cỳlty cǎw nùr tot lâl gás kwãl-hy zine \\
    happy loud bright and song but good-\Neg{} 3\Sg{}.\Inanim{} \\
    happy loud bright and song but good-not it \\
    \glt
    The song is happy, loud, and bright, but it is not good.
\end{exe}

\begin{exe}
    \ex
    cỳlty cǎw nùr tot lâl gás kwãlhy zine.
    \glll
    xwè-ce wùg wán-ce gógó krũx-hły wàn \\
    know-\Pst{} 1\Sg{} play-\Pst{} 3\Pl{}.\Anim{} \Dem{}.\Inanim{}.\Dist{}-\Adj{} game \\
    knew me played them that game \\
    \glt
    I knew the game they were playing.
\end{exe}

\begin{exe}
    \ex
    õgce wánryl gūs zynyt.
    \glll
    õg-ce wán-ryl gūs zynyt \\
    use-\Pst{} play-\Agt{}.\Anim{} small stone \\
    used player small stone \\
    \glt
    The players used small stones.
\end{exe}

(No, tea is not especially important to parrots.)
\begin{exe}
    \ex
    hēwak hàgry él lỳ sōkro kwalty, tot hēwak hàgry él słàn sōkro kwalty.
    \glll
    hēwak hàgry él lỳ sōkro kwal=ty tot hēwak hàgry él słàn sōkro kwal=ty \\
    drink some person green leaf water=\Poss{} and drink some person black leaf water=\Poss{} \\
    drink some person green leaf water=\Poss{} and drink some person black leaf water=\Poss{} \\
    \glt
    Some people drink green tea, and some people drink black tea.
\end{exe}

\begin{exe}
    \ex
    hēwakce wánce kũxkũx wàn kũxhły él lỳ sōkro kwalty.
    \glll
    hēwak-ce wán-ce kũxkũx wàn kũxhły él lỳ sōkro kwal=ty \\
    drink-\Pst{} play-\Pst{} \Dem{}.\Anim{}.\Dist{}.\Pl{} game \Dem{}.\Anim{}.\Dist{}-\Adj{} person green leaf water=\Poss{} \\
    drank played that game those people green leaf water=\Poss{} \\
    \glt
    The people playing the game were drinking green tea.
\end{exe}

\begin{exe}
    \ex
    hēwakce wánce kũxkũx wàn kũxhły él lỳ sōkro kwalty.
    \glll
    hēwak-ce wán-ce kũxkũx wàn kũxhły él lỳ sōkro kwal=ty \\
    drink-\Pst{} play-\Pst{} \Dem{}.\Anim{}.\Dist{}.\Pl{} game \Dem{}.\Anim{}.\Dist{}-\Adj{} person green leaf water=\Poss{} \\
    drank played that game those people green leaf water=\Poss{} \\
    \glt
    The people playing the game were drinking green tea.
\end{exe}

\begin{exe}
    \ex
    tāga rỳt lúk sōkro kwalty kwałŷxtyni.
    \glll
    tāga rỳt lúk sōkro kwal=ty kwałŷx=ty=ni \\
    grow warm place leaf water=\Poss{} tree=\Poss{}=\InessTwo{} \\
    grow warm place leaf water=\Poss{} tree=\Poss{}=in \\
    \glt
    The tea tree grows in warm places.
    [geographically]
\end{exe}

\begin{exe}
    \ex
    tīke slôrslôr rõk ozkāla.
    \glll
    tīke slôrslôr rõk ozkā=la \\
    white \Dem{}.\Inanim{}.\Prox{}.\Pl{} mountain ice=\AdessThree{} \\
    white these mountains ice=on \\
    \glt
    The ice on these mountains is white.
\end{exe}

\begin{exe}
    \ex
    kȳhi sitǎr zèkhły xār tôctyn krârwik kȳt.
    \glll
    kȳhi sitǎr zèk-hły xār tôctyn krârwik kȳt \\
    able show electricity-\Adj{} brain many idea 4\Sg{} \\
    able show electric brain many idea one \\
    \glt
    The computer can show one many ideas.
\end{exe}

\begin{exe}
    \ex
    hłīkwyk sotê hūg slycgaty.
    \glll
    hłīk-wyk sotê hūg slyc-ga=ty \\
    make-\Fut{} rain fire die-\Ger{}=\Poss{} \\
    make.will rain fire dying=\Poss{} \\
    \glt
    The rain will put out the fire.
    /
    The rain will kill the fire.
\end{exe}

\begin{exe}
    \ex
    câwtuce gó kwałŷx xãkty xełla.
    \glll
    câwtu-ce gó kwałŷx xãk=ty xeł=la \\
    find-\Pst{} 3\Sg{}.\Anim{} tree bottom=\Poss{} \Refl{}.\Sg{}=\AdessThree{} \\
    found him tree bottom=\Poss{} self=at \\
    \glt
    He found himself under the tree.
\end{exe}

\begin{exe}
    \ex
    kwǒnce kwałŷx xãkty góla.
    \glll
    kwǒn-ce kwałŷx xãk=ty gó=la \\
    sleep-\Pst{} tree bottom=\Poss{} 3\Sg{}.\Anim{}=\AdessThree{} \\
    slept tree bottom=\Poss{} him=at \\
    \glt
    He slept under the tree.
\end{exe}

\begin{exe}
    \ex
    kwałŷx xãkty crákli.
    \glll
    kwałŷx xãk=ty crák=li \\
    tree bottom=\Poss{} ant=\InessThree{} \\
    tree bottom=\Poss{} ant=in \\
    \glt
    The ant was in the bottom part of the tree.
\end{exe}

\begin{exe}
    \ex
    sît hàgry crák sōkro gás kȳhi kōg gógó crò ǐ \\
    \glll
    sît hàgry crák sōkro gás kȳhi kōg gógó crò ǐ \\
    seek some ant leaf but able eat 3\Pl{}.\Anim{} other thing \\
    seek some ant leaf but able eat them other things \\
    \glt
    Some ants search for leaves but they can eat other things.
\end{exe}

\begin{exe}
    \ex
    cohon crò crák tùl kwal
    \glll
    cohon crò crák tùl kwal \\
    create other ant sweet water \\
    create other ant sweet water. \\
    \glt
    Other ants make sweet water.
\end{exe}

\begin{exe}
    \ex
    xrôk ìl.
    \glll
    xrôk ìl \\
    big language \\
    big language \\
    \glt
    Languages are big.
\end{exe}

\begin{exe}
    \ex
    xwèce zàce kũx xrôłga kũxhły wihǔ kiłìl.
    \glll
    xwè-ce zà-ce kũx xrôł-ga kũx-hły wihǔ kiłìl \\
    know-\Pst{} do-\Pst{} \Dem{}.\Anim{}.\Dist{}.\Sg{} say-\Ger{} \Dem{}.\Anim{}.\Dist{}-\Adj{} bird secret \\
    knew did that saying that bird secret \\
    \glt
    The bird who spoke knew a secret.
\end{exe}

\begin{exe}
    \ex
    gāwce gó zine wihǔ gás gāwcehy gó zine hełǎ.
    \glll
    gāw-ce gó zine wihǔ gás gāw-ce-hy gó zine hełǎ \\
    give-\Pst{} 3\Sg{}.\Anim{} 3\Sg{}.\Inanim{} bird but give-\Pst{}-\Neg{} 3\Sg{}.\Anim{} 3\Sg{}.\Inanim{} human \\
    gave him bird but gave-not him human \\
    \glt
    He gave it to the birds but not the humans.
\end{exe}

\subsection{Noun Phrases}\label{subsec:noun-phrases}

\begin{exe}
    \ex
    \glll
    rõk kwałŷx=la lelō=li \\
    mountain tree=\AdessThree{} nest=\InessThree{} \\
    mountain tree=on nest=in \\
    \glt
    the nest in the tree on the mountain
\end{exe}
This example exhibits nested stance forms.

\begin{exe}
    \ex
    \glll
    cycīl rõk \\
    red mountain \\
    red mountain \\
    \glt
    the red mountain
\end{exe}
Adjectives come before nouns.

\begin{exe}
    \ex
    \glll
    hełǎ ìl \\
    human language=\Poss{} \\
    human language=\Poss{} \\
    \glt
    human language
\end{exe}
Where English uses a noun as an adjective of ownership or origin,
I will just use the possessive form.

\begin{exe}
    \ex
    \glll
    twazwa-hły wihǔ \\
    metal-\Adj{} bird \\
    metal bird \\
    \glt
    airplane
\end{exe}
The -hły suffix turns material nouns into adjectives.

\begin{exe}
    \ex
    \glll
    xrõk gón slôr-hły hlātu=la \\
    amazing food \Dem{}.\Prox{}-\Adj{} plate=\AdessThree{} \\
    amazing food this plate=on \\
    \glt
    the amazing food on this plate
\end{exe}
The -hły suffix also turns demonstrative pronouns into adjectives.
This causes ambiguity;
the above sentence also could mean
``the amazing food on the plate made of this [material]''.

\begin{exe}
    \ex
    \glll
    wùg hengó=ty \\
    1\Sg{} forest=\Poss{} \\
    I forest=\Poss{} \\
    \glt
    my forest
\end{exe}

\begin{exe}
    \ex
    \glll
    gógó lutlùt gūs crizǐ=na=ty \\
    3\Pl{} river small house=\AdessTwo{}=\Poss{}  \\
    they river small house=by=\Poss{} \\
    \glt
    their small house by the river
\end{exe}
I decided ``their'' modifies ``small house by the river'' rather than just ``small house''.

\begin{exe}
    \ex
    \glll
    xeł krârwik=ty  \\
    \Refl{}.\Sg{} idea=\Poss{} \\
    self idea=\Poss{} \\
    \glt
    one's own idea
\end{exe}
Without an antecedent, this could mean
``my own idea'',
``your own idea'',
``its own idea'',
or
``one's own idea''.
Reflexive possessives don't translate into English perfectly,
so as an example,
it would be used in
``He disliked his idea'',
where ``his'' refers to the subject rather than someone else.

\begin{exe}
    \ex
    \glll
    kȳt xwé=ty gòn=i \\
    4\Sg{} knowledge=\Poss{} faith=\IllThree{} \\
    one knowledge=\Poss{} faith=into \\
    \glt
    faith in one's knowledge
\end{exe}
I decided to use 3D ``into'' motion arbitrarily.
English idioms with prepositions
will have different positional relations.

\begin{exe}
    \ex
    \glll
    gó crizǐ=ty hòk=ty xrôk kwałŷx=li gūs cycīl zêg=la \\
    he house=\Poss{} back=\Poss{} big tree=\InessThree{} small red fruit=\AdessThree{} \\
    he house=\Poss{} back=\Poss{} big tree=in small red fruit=on \\
    \glt
    the small red fruit on the big tree behind his house
\end{exe}

\begin{exe}
    \ex
    \glll
    hàgry wùg wuzic ǐ=ty \\
    some 1\Sg{} favorite thing=\Poss{} \\
    some me favorite thing=\Poss{} \\
    \glt
    some of my favorite things
\end{exe}
In English, ``some'' is a determiner with the associated special behavior.
In Psittacine, I choose for ``some'' to just be a normal adjective.
This particular usage of ``some'' refers to multiple instances of a countable noun,
rather than a portion of a mass noun.

\begin{exe}
    \ex
    \glll
    sotê zasõg=ty \\
    rain season=\Poss{} \\
    rain season=\Poss{} \\
    \glt
    spring
\end{exe}
I will use possessive and stance forms
rather than true compound words.

\begin{exe}
    \ex
    \glll
    krũx-hły kàt nes=ni \\
    \Dem{}.\Dist{}-\Adj{} day event=\InessTwo{} \\
    that day event=in \\
    \glt
    an event on that day
\end{exe}
I only equipped Psittacine with
two-dimensional and three-dimensional positionals.
I decided to use the two-dimensional positionals
for time even though time only has one dimension.

\begin{exe}
    \ex
    \glll
    krũx-hły kàt xâr=ty nes=na \\
    \Dem{}.\Dist{}-\Adj{} day top=\Poss{} event=\AdessTwo{} \\
    that day top=\Poss{} event=on \\
    \glt
    an event before that day
\end{exe}
I decided to make the past upwards and the future downwards.

\begin{exe}
    \ex
    \glll
    kłǎs-hły nūr \\
    glass-\Adj{} light \\
    glass light \\
    \glt
    lightbulb
\end{exe}

\begin{exe}
    \ex
    \glll
    nũr-hły sōkro kláte=ni \\
    feather-\Adj{} leaf design=\InessTwo{} \\
    feathery leaf design=on \\
    \glt
    the design on the cloth
\end{exe}

\begin{exe}
    \ex
    \glll
    xǎw nī \\
    young cat \\
    young cat \\
    \glt
    the kitten
\end{exe}
I decided to prefer the adjectives ``small'' and ``young'' over diminuitive morphology.

\begin{exe}
    \ex
    \glll
    héz nī=li \\
    hat cat=\InessThree{} \\
    hat cat=in \\
    \glt
    the cat in the hat
\end{exe}
(Here, ``in'' means contained in, not wearing.)

\begin{exe}
    \ex
    \glll
    tòg takís=ni \\
    east country=\InessTwo{} \\
    east country=in \\
    \glt
    the country in the east
\end{exe}

\begin{exe}
    \ex
    \glll
    kłǎs-hły hlōs \\
    glass-\Adj{} screen \\
    glass screen \\
    \glt
    window
\end{exe}

\begin{exe}
    \ex
    \glll
    wug gó amì=ty=keg \\
    dog him friend=\Poss{}=\Com{} \\
    dog him friend=\Poss{}=with \\
    \glt
    his friend with the dog
\end{exe}

\begin{exe}
    \ex
    \glll
    gógó kîr=ty \\
    3\Pl{} heart=\Poss{} \\
    them heart=\Poss{} \\
    \glt
    their hearts
\end{exe}
(``they'' is animate here.)

\begin{exe}
    \ex
    \glll
    hūg-hły kîr wihǔ=keg \\
    fire-\Adj{} heart bird=\Com{} \\
    fiery heart bird=with \\
    \glt
    angry birds
\end{exe}
I chose to express emotion using adjectives on ``heart'',
which is common cross-linguistically.
I expect parrots to have emotional responses with heart rate
in a similar way to humans.

\begin{exe}
    \ex
    \glll
    zine hengó=ni \\
    it forest=\InessTwo{} \\
    it forest=in \\
    \glt
    the forest in it [a country or area]
\end{exe}

\begin{exe}
    \ex
    \glll
    zinezine rõk=ty lutlùt=na \\
    them mountain=\Poss{} river=\AdessTwo{} \\
    them mountain=\Poss{} river=by \\
    \glt
    the river by their mountain
\end{exe}
``they'' is inanimate here.
Since the positional is two-dimensional,
it means ``on the edge of the mountain as a region''
rather than ``on the surface of the mountain''.

\begin{exe}
    \ex
    \glll
    xwẽ crizǐ=ty \\
    study house=\Poss{} \\
    study house=\Poss{} \\
    \glt
    the school
\end{exe}

\begin{exe}
    \ex
    \glll
    zīg sotê kwal=ty \\
    fresh rain water=\Poss{} \\
    fresh rain water=\Poss{} \\
    \glt
    the fresh rainwater
\end{exe}
This could also be bracketed as ``the fresh rain's water''.

\begin{exe}
    \ex
    \glll
    hàgry zine sōkro=ty \\
    some it leaf=\Poss{} \\
    some it leaf=\Poss{} \\
    \glt
    some of its leaves
\end{exe}

\begin{exe}
    \ex
    \glll
    xàc hitza kàt=ken \\
    next pizza day=\Com{} \\
    next pizza day=with \\
    \glt
    the next day with pizza
\end{exe}
This could also be bracketed as ``the day with the next pizza''.

\begin{exe}
    \ex
    \glll
    à nes crò kwãl krârwik=i=ty \\
    2\Sg{} event other good idea=\IllThree{}=\Poss{} \\
    you event other good idea=into=\Poss{} \\
    \glt
    your other good idea about the event
\end{exe}
Positional phrases often don't correspond with the English prepositions
when the usage is not physical.
In this case, ideas are ``into'' their topics.

\begin{exe}
    \ex
    \glll
    wùgwùg xwẽ crizǐ=ty=ty gān=ty gón crizǐ=ty=ni \\
    1\Pl{} study house=\Poss{}=\Poss{} south=\Poss{} food house=\Poss{}=\InessTwo{} \\
    us study house=\Poss{}=\Poss{} south=\Poss{} food house=\Poss{}=in \\
    \glt
    the restaurant south of our school
\end{exe}

\begin{exe}
    \ex
    \glll
    kȳt él=ken \\
    4\Sg{} people=\Com{} \\
    one people=with \\
    \glt
    the people with one
\end{exe}

\begin{exe}
    \ex
    \glll
    krũx-hły rùc rõk=ken \\
    \Dem{}.\Dist{}-\Adj{} gold mountain=\Com{} \\
    that gold mountain=with \\
    \glt
    the mountain with that gold
\end{exe}

\begin{exe}
    \ex
    \glll
    krũx-hły xu \\
    \Dem{}.\Dist{}-\Adj{} shoe \\
    that shoe \\
    \glt
    those shoes
\end{exe}

\begin{exe}
    \ex
    \glll
    zèk-hły xār xwẽ=ty \\
    electricity-\Adj{} brain study=\Poss{} \\
    electric brain study=\Poss{} \\
    \glt
    computer science
\end{exe}

\begin{exe}
    \ex
    \glll
    ìl xwẽ=ty \\
    language study=\Poss{} \\
    language study=\Poss{} \\
    \glt
    linguistics
\end{exe}

\begin{exe}
    \ex
    \glll
    xwé crizǐ=ty zū sōkro=ty \\
    knowledge house=\Poss{} thick leaf=\Poss{} \\
    knowledge house=\Poss{} thick leaf=\Poss{} \\
    \glt
    library card
\end{exe}

\begin{exe}
    \ex
    \glll
    nūr-hły kláte \\
    light-\Adj{} design \\
    light design \\
    \glt
    [digital] image
\end{exe}

\begin{exe}
    \ex
    \glll
    kwałŷx gón=ty  \\
    tree food=\Poss{} \\
    tree food=\Poss{} \\
    \glt
    fertilizer
\end{exe}

\begin{exe}
    \ex
    \glll
    gógó gò kwal=ty=ty \\
    3\Pl{} fruit water=\Poss{}=\Poss{} \\
    them fruit water=\Poss{}=\Poss{} \\
    \glt
    their juice
\end{exe}

\begin{exe}
    \ex
    \glll
    crizǐ kwałŷx=na gò=ty \\
    house tree=\AdessTwo{} fruit=\Poss{} \\
    house tree=on fruit=\Poss{} \\
    \glt
    the trees by the house's fruit
\end{exe}

\begin{exe}
    \ex
    \glll
    hàgry wùg wuzic gò=ty \\
    some me favorite fruit=\Poss{} \\
    some me favorite fruit=\Poss{} \\
    \glt
    some of my favorite fruit
\end{exe}
Specifically, several pieces of my favorite type of fruit.

\begin{exe}
    \ex
    \glll
    wùg hàgry wuzic gò=ty \\
    me some favorite fruit=\Poss{} \\
    me some favorite fruit=\Poss{} \\
    \glt
    some of my favorite fruit
\end{exe}
Specifically, several types of fruit.

\begin{exe}
    \ex
    \glll
    słarý \\
    grass \\
    grass \\
    \glt
    grass
\end{exe}

\begin{exe}
    \ex
    \glll
    kwałŷx słarý=ty \\
    tree grass=\Poss{} \\
    tree grass=\Poss{} \\
    \glt
    moss
\end{exe}

\begin{exe}
    \ex
    \glll
    słarý-hły kwałŷx \\
    grass-\Adj{} tree \\
    grass tree \\
    \glt
    bamboo
\end{exe}

\begin{exe}
    \ex
    \glll
    kłǎs-hły ǎn \\
    glass-\Adj{} eye \\
    glass eye \\
    \glt
    camera
\end{exe}
(or simply ``glass eye'')

\begin{exe}
    \ex
    \glll
    ǎn kwal=ty  \\
    eye water=\Poss{} \\
    eye water=\Poss{} \\
    \glt
    tear
\end{exe}

\begin{exe}
    \ex
    \glll
    ozkā-hły sotê \\
    ice-\Adj{} rain \\
    icy rain \\
    \glt
    snow
\end{exe}

\begin{exe}
    \ex
    \glll
    tùł ozkā \\
    sweet ice \\
    sweet ice \\
    \glt
    sugar
\end{exe}