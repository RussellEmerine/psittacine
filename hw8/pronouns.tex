\subsection{Pronouns}\label{subsec:pronouns}

Pronouns pluralize by reduplication,
and have an animacy distinction in 3rd person pronouns.
Mass nouns are considered singular.
First person plural can be inclusive or exclusive.
Animals are animate, and everything else is inanimate.
The decision to have pluralization for pronouns
but not for nouns is taken from Chinese.

There is a 4th person ``generic'' pronoun,
with singular used for generic ``you'' or generic ``one''
and plural used for generic ``they''.
There is a single reflexive pronoun
that applies for any person.
It is pluralized to match its antecedent.

There are proximal and distal demonstratives
in animate and inanimate forms.
These are listed as pronouns.
The adjective forms, e.g.\ ``this book'', are just normal adjectives,
which will be derived from the pronoun forms with -hły attached.
The adjective forms do not have plurals.

\begin{center}
    \begin{tabular}{|c|c|c|}
        \hline
        Person                                  & Singular & Plural   \\
        \hline
        1st                                     & wùg      & wùgwùg   \\
        \hline
        2nd                                     & à        & àà       \\
        \hline
        3rd animate                             & gó       & gógó     \\
        \hline
        3rd inanimate                           & zine     & zinezine \\
        \hline
        4th                                     & kȳt      & kȳtkȳt   \\
        \hline
        Reflexive (self)                        & xeł      & xełxeł   \\
        \hline
        Proximal demonstrative animate (this)   & nôr      & nôrnôr   \\
        \hline
        Proximal demonstrative inanimate (this) & slôr     & slôrslôr \\
        \hline
        Distal demonstrative animate (that)     & kũx      & kũxkũx   \\
        \hline
        Distal demonstrative inanimate (that)   & krũx     & krũxkrũx \\
        \hline
    \end{tabular}
\end{center}

\subsection{Examples}\label{subsec:examples}

The following are a few examples of pronoun usage
that are less obvious.

\begin{exe}
    \ex
    \glt
    kȳt xwéty gòni
    \glll
    kȳt xwé=ty gòn=i \\
    4\Sg{} knowledge=\Poss{} faith=\IllThree{} \\
    one knowledge=\Poss{} faith=into \\
    \glt
    faith in one's knowledge
\end{exe}
An instance of a 4th person singular pronoun in use.

\begin{exe}
    \ex
    xrôł kȳtkȳt kazcǒ gó gón
    \glll
    xrôł kȳtkȳt kazcǒ gó gón \\
    say 4\Pl{} want 3\Sg{}.\Anim{} food \\
    say them want him food \\
    \glt
    They say he wants food.
\end{exe}
An instance of a 4th person plural pronoun in use.
The English phrase ``They say \dots'' is the most typical
kind of usage for 4th person plural.

\begin{exe}
    \ex
    \glll
    xeł krârwik=ty  \\
    \Refl{}.\Sg{} idea=\Poss{} \\
    self idea=\Poss{} \\
    \glt
    one's own idea
\end{exe}
An example of a reflexive pronoun in use.
Without an antecedent, this could mean
``my own idea'',
``your own idea'',
``its own idea'',
or
``one's own idea''.
Reflexive possessives don't translate into English perfectly,
so as an example,
it would be used in
``He disliked his idea'',
where ``his'' refers to the subject rather than someone else.

