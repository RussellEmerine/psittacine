\documentclass[12pt]{article}
\pagestyle{plain}
\usepackage[letterpaper,margin=1in]{geometry}
\usepackage{tipa}
\usepackage{amsmath}
\usepackage{leipzig}
\usepackage{gb4e}

\newleipzig{InessTwo}{iness2}{2d inessive}
\newleipzig{InessThree}{iness3}{3d inessive}
\newleipzig{AdessTwo}{adess2}{2d adessive}
\newleipzig{AdessThree}{adess3}{3d adessive}
\newleipzig{IllThree}{ill3}{3d illative}
\newleipzig{AllTwo}{all2}{2d allative}
\newleipzig{AllThree}{all3}{3d allative}
\newleipzig{AblThree}{abl3}{3d ablative}

\makeglossaries

% Document
\begin{document}
    \subsubsection*{Mechanics}

    Stance forms (taking the place of prepositional phrases and possession)
    are described in the previous homework.
    Definiteness, gender, number, and case are not marked.

    Adjectives are placed before the noun,
    with no modifications to the noun or adjective.

    The -hły suffix will mark adjectivization.
    The most important use of -hły will be turning material nouns into adjectives.

    \subsubsection*{Pronouns}

    I chose to pluralize pronouns by reduplication,
    and have an animacy distinction in 3rd person pronouns.
    Mass nouns are considered singular.
    First person plural can be inclusive or exclusive.
    Animals are animate, and everything else is inanimate.
    The decision to have pluralization for pronouns
    but not for nouns is taken from Chinese.

    I chose to have a 4th person ``generic'' pronoun,
    with singular used for generic ``you'' or generic ``one''
    and plural used for generic ``they''.
    (I plan to also use this for passive voice.)

    I chose to have a single reflexive pronoun
    that applies for any person.
    It is pluralized to match its antecedent.

    I chose to have proximal and distal demonstratives
    in animate and inanimate forms.
    These are listed as pronouns.
    The adjective forms, e.g.\ ``this book'', are just normal adjectives,
    which will be derived from the pronoun forms with -hły attached.
    The adjective forms do not have plurals.

    \begin{center}
        \begin{tabular}{|c|c|c|}
            \hline
            Person                                  & Singular & Plural   \\
            \hline
            1st                                     & wùg      & wùgwùg   \\
            \hline
            2nd                                     & à        & àà       \\
            \hline
            3rd animate                             & gó       & gógó     \\
            \hline
            3rd inanimate                           & zine     & zinezine \\
            \hline
            4th                                     & kȳt      & kȳtkȳt   \\
            \hline
            Reflexive (self)                        & xeł      & xełxeł   \\
            \hline
            Proximal demonstrative animate (this)   & nôr      & nôrnôr   \\
            \hline
            Proximal demonstrative inanimate (this) & slôr     & slôrslôr \\
            \hline
            Distal demonstrative animate (that)     & kũx      & kũxkũx   \\
            \hline
            Distal demonstrative inanimate (that)   & krũx     & krũxkrũx \\
            \hline
        \end{tabular}
    \end{center}

    \subsubsection*{Noun Phrases}

    \begin{exe}
        \ex
        \glll
        rõk kwałŷx=la lelō=li \\
        mountain tree=\AdessThree{} nest=\InessThree{} \\
        mountain tree=on nest=in \\
        \glt
        the nest in the tree on the mountain
    \end{exe}
    This example exhibits nested stance forms.

    \begin{exe}
        \ex
        \glll
        cycīl rõk \\
        red mountain \\
        red mountain \\
        \glt
        the red mountain
    \end{exe}
    Adjectives come before nouns.

    \begin{exe}
        \ex
        \glll
        hełǎ ìl \\
        human language=\Poss{} \\
        human language=\Poss{} \\
        \glt
        human language
    \end{exe}
    Where English uses a noun as an adjective of ownership or origin,
    I will just use the possessive form.

    \begin{exe}
        \ex
        \glll
        twazwa-hły wihǔ \\
        metal-\Adj{} bird \\
        metal bird \\
        \glt
        airplane
    \end{exe}
    The -hły suffix turns material nouns into adjectives.

    \begin{exe}
        \ex
        \glll
        xrõk gón slôr-hły hlātu=la \\
        amazing food \Dem{}.\Prox{}-\Adj{} plate=\AdessThree{} \\
        amazing food this plate=on \\
        \glt
        the amazing food on this plate
    \end{exe}
    The -hły suffix also turns demonstrative pronouns into adjectives.
    This causes ambiguity;
    the above sentence also could mean
    ``the amazing food on the plate made of this [material]''.

    \begin{exe}
        \ex
        \glll
        wùg hengó=ty \\
        1\Sg{} forest=\Poss{} \\
        I forest=\Poss{} \\
        \glt
        my forest
    \end{exe}

    \begin{exe}
        \ex
        \glll
        gógó lutlùt gūs crizǐ=na=ty \\
        3\Pl{} river small house=\AdessTwo{}=\Poss{}  \\
        they river small house=by=\Poss{} \\
        \glt
        their small house by the river
    \end{exe}
    I decided ``their'' modifies ``small house by the river'' rather than just ``small house''.

    \begin{exe}
        \ex
        \glll
        xeł krârwik=ty  \\
        \Refl{}.\Sg{} idea=\Poss{} \\
        self idea=\Poss{} \\
        \glt
        one's own idea
    \end{exe}
    Without an antecedent, this could mean
    ``my own idea'',
    ``your own idea'',
    ``its own idea'',
    or
    ``one's own idea''.
    Reflexive possessives don't translate into English perfectly,
    so as an example,
    it would be used in
    ``He disliked his idea'',
    where ``his'' refers to the subject rather than someone else.

    \begin{exe}
        \ex
        \glll
        kȳt xwé=ty gòn=i \\
        4\Sg{} knowledge=\Poss{} faith=\IllThree{} \\
        one knowledge=\Poss{} faith=into \\
        \glt
        faith in one's knowledge
    \end{exe}
    I decided to use 3D ``into'' motion arbitrarily.
    English idioms with prepositions
    will have different positional relations.

    \begin{exe}
        \ex
        \glll
        gó crizǐ=ty hòk=ty xrôk kwałŷx=li gūs cycīl zêg=la \\
        he house=\Poss{} back=\Poss{} big tree=\InessThree{} small red fruit=\AdessThree{} \\
        he house=\Poss{} back=\Poss{} big tree=in small red fruit=on \\
        \glt
        the small red fruit on the big tree behind his house
    \end{exe}

    \begin{exe}
        \ex
        \glll
        hàgry wùg wuzic ǐ=ty \\
        some 1\Sg{} favorite thing=\Poss{} \\
        some me favorite thing=\Poss{} \\
        \glt
        some of my favorite things
    \end{exe}
    In English, ``some'' is a determiner with the associated special behavior.
    In Psittacine, I choose for ``some'' to just be a normal adjective.
    This particular usage of ``some'' refers to multiple instances of a countable noun,
    rather than a portion of a mass noun.

    \begin{exe}
        \ex
        \glll
        sotê zasõg=ty \\
        rain season=\Poss{} \\
        rain season=\Poss{} \\
        \glt
        spring
    \end{exe}
    I will use possessive and stance forms
    rather than true compound words.

    \begin{exe}
        \ex
        \glll
        krũx-hły kàt nes=ni \\
        \Dem{}.\Dist{}-\Adj{} day event=\InessTwo{} \\
        that day event=in \\
        \glt
        an event on that day
    \end{exe}
    I only equipped Psittacine with
    two-dimensional and three-dimensional positionals.
    I decided to use the two-dimensional positionals
    for time even though time only has one dimension.

    \begin{exe}
        \ex
        \glll
        krũx-hły kàt xâr=ty nes=na \\
        \Dem{}.\Dist{}-\Adj{} day top=\Poss{} event=\AdessTwo{} \\
        that day top=\Poss{} event=on \\
        \glt
        an event before that day
    \end{exe}
    I decided to make the past upwards and the future downwards.

    \begin{exe}
        \ex
        \glll
        kłǎs-hły nūr \\
        glass-\Adj{} light \\
        glass light \\
        \glt
        lightbulb
    \end{exe}

    \begin{exe}
        \ex
        \glll
        nũr-hły sōkro kláte=ni \\
        feather-\Adj{} leaf design=\InessTwo{} \\
        feathery leaf design=on \\
        \glt
        the design on the cloth
    \end{exe}

    \begin{exe}
        \ex
        \glll
        xǎw nī \\
        young cat \\
        young cat \\
        \glt
        the kitten
    \end{exe}
    I decided to prefer the adjectives ``small'' and ``young'' over diminuitive morphology.

    \begin{exe}
        \ex
        \glll
        héz nī=li \\
        hat cat=\InessThree{} \\
        hat cat=in \\
        \glt
        the cat in the hat
    \end{exe}
    (Here, ``in'' means contained in, not wearing.)

    \begin{exe}
        \ex
        \glll
        tòg takís=ni \\
        east country=\InessTwo{} \\
        east country=in \\
        \glt
        the country in the east
    \end{exe}

    \begin{exe}
        \ex
        \glll
        kłǎs-hły hlōs \\
        glass-\Adj{} screen \\
        glass screen \\
        \glt
        window
    \end{exe}

    \begin{exe}
        \ex
        \glll
        wug gó amì=ty=keg \\
        dog him friend=\Poss{}=\Com{} \\
        dog him friend=\Poss{}=with \\
        \glt
        his friend with the dog
    \end{exe}

    \begin{exe}
        \ex
        \glll
        gógó kîr=ty \\
        3\Pl{} heart=\Poss{} \\
        them heart=\Poss{} \\
        \glt
        their hearts
    \end{exe}
    (``they'' is animate here.)

    \begin{exe}
        \ex
        \glll
        hūg-hły kîr wihǔ=keg \\
        fire-\Adj{} heart bird=\Com{} \\
        fiery heart bird=with \\
        \glt
        angry birds
    \end{exe}
    I chose to express emotion using adjectives on ``heart'',
    which is common cross-linguistically.
    I expect parrots to have emotional responses with heart rate
    in a similar way to humans.

    \begin{exe}
        \ex
        \glll
        zine hengó=ni \\
        it forest=\InessTwo{} \\
        it forest=in \\
        \glt
        the forest in it [a country or area]
    \end{exe}

    \begin{exe}
        \ex
        \glll
        zinezine rõk=ty lutlùt=na \\
        them mountain=\poss{} river=\adesstwo{} \\
        them mountain=\poss{} river=by \\
        \glt
        the river by their mountain
    \end{exe}
    ``they'' is inanimate here.
    Since the positional is two-dimensional,
    it means ``on the edge of the mountain as a region''
    rather than ``on the surface of the mountain''.

    \begin{exe}
        \ex
        \glll
        xwẽ crizǐ=ty \\
        study house=\Poss{} \\
        study house=\Poss{} \\
        \glt
        the school
    \end{exe}

    \begin{exe}
        \ex
        \glll
        zīg sotê kwal=ty \\
        fresh rain water=\Poss{} \\
        fresh rain water=\Poss{} \\
        \glt
        the fresh rainwater
    \end{exe}
    This could also be bracketed as ``the fresh rain's water''.

    \begin{exe}
        \ex
        \glll
        hàgry zine sōkro=ty \\
        some it leaf=\Poss{} \\
        some it leaf=\Poss{} \\
        \glt
        some of its leaves
    \end{exe}

    \begin{exe}
        \ex
        \glll
        xàc hitza kàt=ken \\
        next pizza day=\Com{} \\
        next pizza day=with \\
        \glt
        the next day with pizza
    \end{exe}
    This could also be bracketed as ``the day with the next pizza''.

    \begin{exe}
        \ex
        \glll
        à nes crò kwãl krârwik=i=ty \\
        2\Sg{} event other good idea=\IllThree{}=\Poss{} \\
        you event other good idea=into=\Poss{} \\
        \glt
        your other good idea about the event
    \end{exe}
    Positional phrases often don't correspond with the English prepositions
    when the usage is not physical.
    In this case, ideas are ``into'' their topics.

    \begin{exe}
        \ex
        \glll
        wùgwùg xwẽ crizǐ=ty=ty gān=ty gón crizǐ=ty=ni \\
        1\Pl{} study house=\Poss{}=\Poss{} south=\Poss{} food house=\Poss{}=\InessTwo{} \\
        us study house=\Poss{}=\Poss{} south=\Poss{} food house=\Poss{}=in \\
        \glt
        the restaurant south of our school
    \end{exe}

    \begin{exe}
        \ex
        \glll
        kȳt él=ken \\
        4\Sg{} people=\Com{} \\
        one people=with \\
        \glt
        the people with one
    \end{exe}

    \begin{exe}
        \ex
        \glll
        krũx-hły rùc rõk=ken \\
        \Dem{}.\Dist{}-\Adj{} gold mountain=\Com{} \\
        that gold mountain=with \\
        \glt
        the mountain with that gold
    \end{exe}

    \begin{exe}
        \ex
        \glll
        krũx-hły xu \\
        \Dem{}.\Dist{}-\Adj{} shoe \\
        that shoe \\
        \glt
        those shoes
    \end{exe}

    \begin{exe}
        \ex
        \glll
        zèk-hły xār xwẽ=ty \\
        electricity-\Adj{} brain study=\Poss{} \\
        electric brain study=\Poss{} \\
        \glt
        computer science
    \end{exe}

    \begin{exe}
        \ex
        \glll
        ìl xwẽ=ty \\
        language study=\Poss{} \\
        language study=\Poss{} \\
        \glt
        linguistics
    \end{exe}

    \begin{exe}
        \ex
        \glll
        xwé crizǐ=ty zū sōkro=ty \\
        knowledge house=\Poss{} thick leaf=\Poss{} \\
        knowledge house=\Poss{} thick leaf=\Poss{} \\
        \glt
        library card
    \end{exe}

    \begin{exe}
        \ex
        \glll
        nūr-hły kláte \\
        light-\Adj{} design \\
        light design \\
        \glt
        [digital] image
    \end{exe}

    \begin{exe}
        \ex
        \glll
        kwałŷx gón=ty  \\
        tree food=\Poss{} \\
        tree food=\Poss{} \\
        \glt
        fertilizer
    \end{exe}

    \begin{exe}
        \ex
        \glll
        gógó gò kwal=ty=ty \\
        3\Pl{} fruit water=\Poss{}=\Poss{} \\
        them fruit water=\Poss{}=\Poss{} \\
        \glt
        their juice
    \end{exe}

    \begin{exe}
        \ex
        \glll
        crizǐ kwałŷx=na gò=ty \\
        house tree=\AdessTwo{} fruit=\Poss{} \\
        house tree=on fruit=\Poss{} \\
        \glt
        the trees by the house's fruit
    \end{exe}

    \begin{exe}
        \ex
        \glll
        hàgry wùg wuzic gò=ty \\
        some me favorite fruit=\Poss{} \\
        some me favorite fruit=\Poss{} \\
        \glt
        some of my favorite fruit
    \end{exe}
    Specifically, several pieces of my favorite type of fruit.

    \begin{exe}
        \ex
        \glll
        wùg hàgry wuzic gò=ty \\
        me some favorite fruit=\Poss{} \\
        me some favorite fruit=\Poss{} \\
        \glt
        some of my favorite fruit
    \end{exe}
    Specifically, several types of fruit.

    \begin{exe}
        \ex
        \glll
        słarý \\
        grass \\
        grass \\
        \glt
        grass
    \end{exe}

    \begin{exe}
        \ex
        \glll
        kwałŷx słarý=ty \\
        tree grass=\Poss{} \\
        tree grass=\Poss{} \\
        \glt
        moss
    \end{exe}

    \begin{exe}
        \ex
        \glll
        słarý-hły kwałŷx \\
        grass-\Adj{} tree \\
        grass tree \\
        \glt
        bamboo
    \end{exe}

    \begin{exe}
        \ex
        \glll
        kłǎs-hły ǎn \\
        glass-\Adj{} eye \\
        glass eye \\
        \glt
        camera
    \end{exe}
    (or simply ``glass eye'')

    \begin{exe}
        \ex
        \glll
        ǎn kwal=ty  \\
        eye water=\Poss{} \\
        eye water=\Poss{} \\
        \glt
        tear
    \end{exe}

    \begin{exe}
        \ex
        \glll
        ozkā-hły sotê \\
        ice-\Adj{} rain \\
        icy rain \\
        \glt
        snow
    \end{exe}

    \begin{exe}
        \ex
        \glll
        tùł ozkā \\
        sweet ice \\
        sweet ice \\
        \glt
        sugar
    \end{exe}

\end{document}