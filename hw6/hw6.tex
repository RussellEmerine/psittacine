\documentclass[12pt]{article}
\pagestyle{plain}
\usepackage[letterpaper,margin=1in]{geometry}
\usepackage{tipa}
\usepackage{amsmath}
\usepackage{leipzig}
\usepackage{gb4e}

\newleipzig{InessTwo}{iness2}{2d inessive}
\newleipzig{InessThree}{iness3}{3d inessive}
\newleipzig{AdessTwo}{adess2}{2d adessive}
\newleipzig{AdessThree}{adess3}{3d adessive}
\newleipzig{IllThree}{ill3}{3d illative}
\newleipzig{AllTwo}{all2}{2d allative}
\newleipzig{AllThree}{all3}{3d allative}
\newleipzig{AblThree}{abl3}{3d ablative}
\newleipzig{Ger}{ger}{gerund}
\newleipzig{Anim}{anim}{animate}
\newleipzig{Inanim}{inanim}{inanimate}
\newleipzig{Hst}{hst}{historical}
\newleipzig{Int}{int}{interrogative}
\newleipzig{Agt}{agt}{agentive}

\makeglossaries

% Document
\begin{document}
    \subsubsection*{Word Order}

    The typical word order is
    \begin{center}
        \quad Verb \quad Subject \quad [\quad Direct Object \quad  [ \quad Indirect Object \quad ] \quad ]
    \end{center}
    where each verb takes a specific number of arguments.
    The language is strictly nominative-accusative.
    Word order is the only indication of which noun is the subject, direct object, or indirect object.

    \begin{exe}
        \ex
        \glt
        kwǒn wihǔ.
        \glll
        kwǒn wihǔ \\
        sleep bird \\
        sleep bird \\
        \glt
        The bird sleeps.
    \end{exe}

    \begin{exe}
        \ex
        \glt
        krân wùg wug.
        \glll
        krân wùg wug \\
        see 1\Sg{} dog \\
        see me dog \\
        \glt
        I see the dog.
    \end{exe}

    \begin{exe}
        \ex
        \glt
        krân wug wùg.
        \glll
        krân wug wùg \\
        see dog 1\Sg{} \\
        see dog me \\
        \glt
        The dog sees me.
    \end{exe}

    \begin{exe}
        \ex
        \glt
        gāw gó hlātu wùg.
        \glll
        gāw gó hlātu wùg \\
        give 3\Sg{}.\Anim{} plate 1\Sg{} \\
        give him plate me \\
        \glt
        He gives me a plate.
    \end{exe}

    \subsubsection*{Auxiliary Verbs}

    Verbs have an intrinsic valency.
    Some are intransitive, some are transitive, and a few are ditransitive.
    The language uses auxiliary verbs with gerunds to change transitivity.

    Gerunds are formed by adding the suffix \textit{-ga} to a verb.

    The auxiliary verb \textit{zà} ``do''
    can take a transitive or ditransitive gerund as an object
    and produce an intransitive form.
    \textit{zà} is used exclusively for changing valency
    and cannot be used for sentences like ``I do the homework''.

    \begin{exe}
        \ex
        \glt
        zà wihǔ kōgga.
        \glll
        zà wihǔ kōg-ga \\
        do bird eat-\Ger{} \\
        do bird eating \\
        \glt
        The bird eats.
    \end{exe}

    \begin{exe}
        \ex
        \glt
        zà kwałŷx gāwga.
        \glll
        zà kwałŷx gāw-ga \\
        do tree give-\Ger{} \\
        do tree giving \\
        \glt
        The tree gives.
        /
        The tree provides.
    \end{exe}

    The auxiliary verb \textit{hłĩk} ``make, cause''
    can take a gerund as an object.
    If the gerund is possessed by a noun,
    the noun is the patient of the sentence.
    The causative formed is the most general type of causative,
    where the patient is not necessarily forced.

    \begin{exe}
        \ex
        \glt
        hłĩk howeg kwałŷx xrîzgaty.
        \glll
        hłĩk howeg kwałŷx xrîz-ga=ty \\
        make wind tree shake-\Ger{}=\Poss{} \\
        make wind tree shaking=\Poss{} \\
        \glt
        The wind makes the tree shake.
        /
        The wind shakes the tree.
    \end{exe}

    \begin{exe}
        \ex
        \glt
        hłĩk howeg xrîzga.
        \glll
        hłĩk howeg xrîz-ga \\
        make wind shake-\Ger{} \\
        make wind shaking \\
        \glt
        The wind shakes [things].
        /
        The wind shakes the tree.
    \end{exe}
    \textit{xrîz} is naturally intransitive.

    \begin{exe}
        \ex
        \glt
        hłîk wihǔ xǎw wihǔ kōggaty.
        \glll
        hłîk wihǔ xǎw wihǔ kōg-ga=ty \\
        make bird small bird eat-\Ger{}=\Poss{} \\
        make bird small bird eating=\Poss{} \\
        \glt
        The bird makes the chick eat.
        /
        The bird feeds the chick.
    \end{exe}

    The auxiliary verb \textit{xal} ``make, cause'',
    similarly to \textit{hłĩk},
    takes a gerund as an object.
    However, \textit{xal} also takes in indirect object,
    which acts as the direct object of the original transitive verb.

    \begin{exe}
        \ex
        \glt
        xal wihǔ xǎw wihǔ kōggaty gò.
        \glll
        xal wihǔ xǎw wihǔ kōg-ga=ty gò \\
        make bird small bird eat-\Ger{}=\Poss{} fruit \\
        make bird small bird eating=\Poss{} fruit \\
        \glt
        The bird makes the chick eat fruit.
        /
        The bird feeds the chick fruit.
    \end{exe}

    \subsubsection*{Replacements for be, have, and go}

    The language uses the zero copula.
    This may be used to relate two nouns
    or state a noun in a stance form.
    Because of the language's complex directional system,
    this takes the place of ``to be'', ``to go'', and ``to have''.

    \begin{exe}
        \ex
        \glt
        wùg hełǎ.
        \glll
        wùg hełǎ \\
        1\Sg{} human \\
        me human \\
        \glt
        I am a human.
    \end{exe}

    \begin{exe}
        \ex
        \glt
        zine wùg zyla nũrty.
        \glll
        zine wùg zyla nũr=ty \\
        3\Sg{}.\Inanim{} 1\Sg{} blue feather=\Poss{} \\
        it me blue feather=\Poss{} \\
        \glt
        It's my blue feather.
    \end{exe}

    \begin{exe}
        \ex
        \glt
        wùg zyla nũrty.
        \glll
        wùg zyla nũr=ty \\
        1\Sg{} blue feather=\Poss{} \\
        me blue feather=\Poss{} \\
        \glt
        The blue feather is mine.
        /
        I have a blue feather.
    \end{exe}

    \begin{exe}
        \ex
        \glt
        crizǐ xârty słygizna.
        \glll
        crizǐ xâr=ty słygiz=na \\
        house top=\Poss{} vine=\AdessTwo{} \\
        house top=\Poss{} vine=towards \\
        \glt
        The vine goes up the house.
    \end{exe}

    Statement of existence equivalent to ``there is''
    does not have a unique construction.
    Rather, it is treated as a place having things.
    If the statement of existence is too general
    to be tied to a particular place,
    ``here'' is used by default,
    or if that is ambiguous,
    ``the world'' is used.

    \begin{exe}
        \ex
        \glt
        kwałŷx gòty.
        \glll
        kwałŷx gò=ty \\
        tree fruit=\Poss{} \\
        tree fruit=\Poss{} \\
        \glt
        The tree has fruit.
        /
        There is fruit in the tree.
    \end{exe}

    \begin{exe}
        \ex
        \glt
        kûk crizǐty.
        \glll
        kûk crizǐ=ty \\
        here house=\Poss{} \\
        here house=\Poss{} \\
        \glt
        There are houses here.
        /
        There are houses [in general].
    \end{exe}

    Adjectives act like intransitive verbs.

    \begin{exe}
        \ex
        \glt
        zyla nũr.
        \glll
        zyla nũr \\
        blue feather \\
        blue feather \\
        \glt
        The feather is blue.
    \end{exe}

    \subsubsection*{Negation}
    A verb is negated by adding the suffix \textit{-hy}.

    \begin{exe}
        \ex
        \glt
        lȳthy èt húgni.
        \glll
        lȳt-hy èt húg=ni \\
        fly-\Neg{} night bees=\InessTwo{} \\
        fly-not night bees=at \\
        \glt
        The bees do not fly at night.
    \end{exe}

    \begin{exe}
        \ex
        \glt
        kōghy wùg rãw.
        \glll
        kōg-hy wùg rãw \\
        eat-\Neg{} 1\Sg{} meat \\
        eat-not me meat \\
        \glt
        I do not eat meat.
    \end{exe}

    Since adjectives are like verbs, the suffix can be added directly to an adjective.

    \begin{exe}
        \ex
        \glt
        zine twazwahłyhy.
        \glll
        zine twazwa-hły-hy \\
        3\Sg{}.\Inanim{} metal-\Adj{}-\Neg{} \\
        it metal-not \\
        \glt
        It's not made of metal.
    \end{exe}

    This also applies to adjectives that are not directly acting as verbs.
    The most obvious application of this is in disambiguation,
    but it can also just be a simple description (not quite an antonym, just a lack of a trait).
    \begin{exe}
        \ex
        \glt
        kōghy wihǔ zighy gò.
        \glll
        kōg-hy wihǔ zig-hy gò \\
        eat-\Neg{} bird fresh-\Neg{} fruit \\
        eat-not bird fresh-not fruit \\
        \glt
        The birds do not eat unfresh fruit.
    \end{exe}

    Since there is no verb when relating two nouns
    or stating a noun in a stance form,
    a special verb must be used for negating such a statement.
    \textit{ì} ``be / go'' is used
    with the negation suffix attached.
    \textit{ì} is special in that
    it is neither transitive nor intransitive,
    and accepts either two nouns or one noun in a stance form.

    \begin{exe}
        \ex
        \glt
        ìhy wùg wihǔ.
        \glll
        ì-hy wùg wihǔ \\
        be-\Neg{} 1\Sg{} bird \\
        be-not me bird \\
        \glt
        I am not a bird.
    \end{exe}

    \begin{exe}
        \ex
        \glt
        ìhy èt kûkni twazwahły hūtna.
        \glll
        ì-hy èt kûk=ni twazwa-hły hūt=na \\
        be-\Neg{} night here=\InessTwo{} metal-\Adj{} caterpillar=\AdessTwo{} \\
        be-not night here=at metal caterpillar=towards \\
        \glt
        The trains do not go here at night.
    \end{exe}

    \subsubsection*{Adverbs}
    Adverbs are just adjectives placed before verbs rather than before nouns.

    \begin{exe}
        \ex
        \glt
        cỳłty kwǒn wùg.
        \glll
        cỳłty kwǒn wùg \\
        happy sleep 1\Sg{} \\
        happy sleep me \\
        \glt
        I sleep happily.
    \end{exe}

    \subsubsection*{Tense}
    There are four tenses, present, past, future, and remote past,
    used for example when telling stories.
    The present is unmarked.
    The other tenses are indicated by verbal suffixes.

    \begin{exe}
        \ex
        \glt
        kwǒnce wùg.
        \glll
        kwǒn-ce wùg \\
        sleep-\Pst{} 1\Sg{} \\
        slept me \\
        \glt
        I slept.
    \end{exe}

    \begin{exe}
        \ex
        \glt
        zàcat takís gréga.
        \glll
        zà-cat takís gré-ga \\
        do-\Hst{} country farm-\Ger{} \\
        did country farming \\
        \glt
        The country farmed [once upon a time].
    \end{exe}

    The tense suffixes occur before \textit{-hy}.
    \begin{exe}
        \ex
        \glt
        xwèwykhy wùg agìty zine.
        \glll
        xwè-wyk-hy wùg agì=ty zine \\
        know-\Fut{}-\Neg{} 1\Sg{} friend=\Poss{} 3\Sg{}.\Inanim{} \\
        know-will-not me friend=\Poss{} it \\
        \glt
        My friend will not know it.
    \end{exe}

    ``be'', ``have'', and ``go''
    are usually expressed without a verb in the present tense.
    In the other tenses, they must have a verb.
    Forms which take one noun in a stance form use the same \textit{ì}
    that negation uses.
    However, forms which relate two nouns
    use the suppletive verb \textit{sèt}, which is not used in present tense.

    \begin{exe}
        \ex
        \glt
        sètwyk hełǎ wùg agìty.
        \glll
        sèt-wyk hełǎ wùg agì=ty \\
        be-\Fut{} human 1\Sg{} friend=\Poss{} \\
        be-will human me friend=\Poss{} \\
        \glt
        The human will be my friend.
    \end{exe}

    \begin{exe}
        \ex
        \glt
        ìce lutlùt wùgna.
        \glll
        ì-ce lutlùt wùg=na \\
        be-\Pst{} river 1\Sg{}=\AdessTwo{} \\
        was river me=at \\
        \glt
        I was at the river.
    \end{exe}

    \begin{exe}
        \ex
        \glt
        ìce lutlùt wùgza.
        \glll
        ì-ce lutlùt wùg=za \\
        be-\Pst{} river 1\Sg{}=\AllTwo{} \\
        went river me=to \\
        \glt
        I went to the river.
    \end{exe}

    \begin{exe}
        \ex
        \glt
        ìcat à gónty.
        \glll
        ì-cat à gón=ty \\
        be-\Fut{} 2\Sg{} food=\Poss{} \\
        have-will you food=\Poss{} \\
        \glt
        The food will be yours.
        /
        You will have food.
    \end{exe}

    Adjectives simply take tense suffixes like normal verbs.

    \begin{exe}
        \ex
        \glt
        xǎwcat wùgwùg.
        \glll
        xǎw-cat wùgwùg \\
        young-\Hst{} 1\Pl{} \\
        young-were we \\
        \glt
        We were young [long ago, or in a story].
    \end{exe}

    \subsubsection*{Aspect and Mood}
    Semantic aspect and mood are not indicated grammatically.
    Rather, if they have reason to be expressed, they are just adverbs
    (or sometimes complex constructions if more detail is required).

    Example for progressive aspect.
    \begin{exe}
        \ex
        \glt
        taktak lȳtce wihǔ.
        \glll
        taktak lȳt-ce wihǔ \\
        for.some.time fly-\Pst{} bird \\
        for.some.time flew bird \\
        \glt
        The bird flew for some time.
        /
        The bird was flying.
    \end{exe}

    Example for iterative aspect.
    \begin{exe}
        \ex
        \glt
        ò xenãtce wùg sōkrohły lâl.
        \glll
        ò xenãt-ce wùg sōkro-hły lâl \\
        again read-\Pst{} 1\Sg{} leaf-\Adj{} song \\
        again read me leafy song \\
        \glt
        I read the book again.
        /
        I reread the book.
    \end{exe}

    Example for potential mood.
    Most adverbs don't apply to the copula in a natural way,
    but this is an instance where it can happen.
    Adverbs applied to the null copula just end up at the start of the sentence.
    \begin{exe}
        \ex
        \glt
        cēłta zine rõk.
        \glll
        cēłta zine rõk \\
        might 3\Sg{}.\Inanim{} mountain \\
        might it mountain \\
        \glt
        It might be the mountain.
    \end{exe}

    \subsubsection*{Conjunctions and Conditionals}

    The basic conjunctions are \textit{tot} ``and''
    and \textit{cec} ``or''.
    When joining two items, the conjunction is placed between.
    When joining three or more items,
    the conjunction may be placed between each item
    or may be used just once after all the items.
    Parallel items all undergo any expected inflection.

    \begin{exe}
        \ex
        \glt
        cǎwce nī tot wihǔ.
        \glll
        cǎw-ce nī tot wihǔ \\
        loud-\Pst{} cat and bird \\
        loud-were cat and bird \\
        \glt
        The cat and the bird were loud.
    \end{exe}

    (\textit{kȳhi} is an adverb.)
    \begin{exe}
        \ex
        \glt
        kȳhi à húgty hūtty crákty cec.
        \glll
        kȳhi à húg=ty hūt=ty crák=ty cec \\
        able 2\Sg{} bee=\Poss{} caterpillar=\Poss{} ant=\Poss{} or \\
        able you bee=\Poss{} caterpillar=\Poss{} ant=\Poss{} or \\
        \glt
        You can have the bee, the caterpillar, or the ant.
    \end{exe}

    \begin{exe}
        \ex
        \glt
        lȳtce, toktõkce, slêxce tot wùgwùg.
        \glll
        lȳt-ce toktõk-ce slêx-ce tot wùgwùg \\
        fly-\Pst{} run-\Pst{} swim-\Pst{} and 1\Pl{} \\
        flew ran swam and us \\
        \glt
        We flew, ran, and swam.
    \end{exe}

    This is also how clauses are joined by conjunctions.

    \begin{exe}
        \ex
        \glt
        toktõkce nī, tot gēntyce wug nī.
        \glll
        toktõk-ce nī tot gēnty-ce wug nī \\
        run-\Pst{} cat and follow-\Pst{} dog cat \\
        ran cat and followed dog cat \\
        \glt
        The cat ran, and the dog followed the cat.
    \end{exe}

    \begin{exe}
        \ex
        \glt
        krânce wihǔ nī, krânce nī wug, krânce wug hełǎ tot \\
        \glll
        krân-ce wihǔ nī krân-ce nī wug krân-ce wug hełǎ tot \\
        see-\Pst{} bird cat see-\Pst{} cat dog see-\Pst{} dog human and \\
        saw bird cat saw cat dog saw dog human and \\
        \glt
        The bird saw the cat, the cat saw the dog, and the dog saw the human.
    \end{exe}

    The emphasized forms ``either \dots or''
    and ``both \dots and''
    can be expressed by placing the conjunction between the words and after the list.

    \begin{exe}
        \ex
        \glt
        nácce wùg sōkrohły lâl tot zū sōkro tot.
        \glll
        nác-ce wùg sōkro-hły lâl tot zū sōkro tot \\
        take-\Pst{} 1\Sg{} leaf-\Adj{} song and thick leaf and \\
        took me leafy song and thick leaf and \\
        \glt
        I took both the book and the card.
    \end{exe}

    Conditional compound sentences are formed
    similarly to conjunctive compound sentences,
    by putting the clauses on either side of a linking word.
    In English, the words relating the clauses can occur in various places,
    e.g.\ ``If X, then Y'' vs.\ ``When X, Y'' vs.\ ``X, yet Y''.
    I instead have just one word that is always between the two sides.

    English has special rules for how tenses are expressed under irrealis moods
    (specifically, ``if I were'' is the prescribed standard).
    I choose for tenses to be expressed based only on time.

    \begin{exe}
        \ex
        \glt
        xal à łỹz tāgagaty, xīg kùwixwyk wùg zinezine.
        \glll
        xal à łỹz tāga-ga=ty xīg kùwix-wyk wùg zinezine \\
        make 2\Sg{} flower grow-\Ger{}=\Poss{} if.then buy-\Fut{} 1\Sg{} 3\Pl{}.\Inanim{} \\
        make you flower growing=\Poss{} if.then buy-will me them \\
        \glt
        If you grow flowers, I will buy them.
    \end{exe}

    \begin{exe}
        \ex
        \glt
        nácce gó rõrhły cíntag, wên gǔzce zine lúkty rygi.
        \glll
        nác-ce gó rõr-hły cíntag wên gǔz-ce zine lúk=ty ryg=i \\
        take-\Pst{} 3\Sg{}.\Anim{} gold-\Adj{} statue when.then put-\Pst{} 3\Sg{}.\Inanim{} place=\Poss{} sand=\IllThree{} \\
        took him golden statue when.then put it place=\Poss{} sand=\IllThree{} \\
        \glt
        When he took the golden statue, he put sand in its place.
    \end{exe}

    \subsubsection*{Subordinate Clauses}

    A subordinate clause describing a noun (i.e.\ a relative clause)
    is formed by using the clause in its standard form,
    replacing each referent to the noun with
    the appropriate demonstrative pronoun form of ``that'',
    and then following the clause with
    the determiner form of ``that'' and the noun.
    If the demonstrative pronoun comes right before
    the demonstrative determiner, the pronoun can be dropped.

    \begin{exe}
        \ex
        \glt
        krân wùg xłǒsce à krũxhły rõk.
        \glll
        krân wùg xłǒs-ce à krũx-hły rõk \\
        see me draw-\Pst{} 2\Sg{} \Dem{}.\Dist{}.\Inanim{}-\Adj{} mountain \\
        see me drew you that mountain \\
        \glt
        I see the mountain that you drew.
    \end{exe}

    \begin{exe}
        \ex
        \glt
        zine zīghy krũx kȳt krũxhły kiłìl.
        \glll
        zine zīg-hy krũx kȳt krũx-hły kiłìl \\
        3\Sg{}.\Inanim{} harm-\Neg{} \Dem.\Dist{}.\Inanim{}.\Sg{} 4\Sg{} \Dem.\Dist{}.\Inanim{}-\Adj{} secret \\
        it harm-not that one secret \\
        \glt
        It is a secret that does not harm one.
    \end{exe}

    (I translate \textit{sît} as ``seek'' in the gloss since it is transitive,
    but as ``search'' in the translation since that is the translation
    that is more faithful to meaning.)
    \begin{exe}
        \ex
        \glt
        câwtuce zàce kũx sîtga kũxhły húg łỹz.
        \glll
        câwtu-ce zà-ce kũx sît-ga kũx-hły húg łỹz \\
        find-\Pst{} do-\Pst{} \Dem{}.\Dist{}.\Anim{}.\Sg{} seek-\Ger{} \Dem{}.\Dist{}.\Anim{}-\Adj{} bee flower \\
        found did that seeking that bee flower \\
        \glt
        The bee that searched found the flower.
    \end{exe}

    Instrumentals are expressed using this form.
    To say ``A did B with C'',
    use ``A that use[d] C did B'' (in the approprate tense).

    \begin{exe}
        \ex
        \glt
        tèkce krũx kwal õgce krũxhły lutlùt rõk.
        \glll
        tèk-ce krũx kwal õg-ce krũx-hły lutlùt rõk \\
        hit-\Pst{} \Dem{}.\Dist{}.\Inanim{}.\Sg{} use-\Pst{} water \Dem{}.\Dist{}.\Inanim{}-\Adj{} river mountain \\
        hit that used water that river mountain \\
        \glt
        The river hit the mountain with water.
    \end{exe}

    A subordinate clause acting as a noun
    is expressed the same way,but just with the demonstrative pronoun
    at the end, rather than the demonstrative determiner and a noun.

    \begin{exe}
        \ex
        câwtuce wùg sîtce wùg krũx.
        \glll
        câwtu-ce wùg sît-ce wùg krũx \\
        find-\Pst{} 1\Sg{} seek-\Pst{} 1\Sg{} \Dem{}.\Inanim{}.\Dist{}.\Sg{} \\
        found me sought me that  \\
        \glt
        I found what I was searching for.
    \end{exe}

    \begin{exe}
        \ex
        nôrhły él hēwakce kũx kwal kũx.
        \glll
        nôr-hły él hēwak-ce kũx kwal kũx \\
        \Dem{}.\Anim{}.\Prox{}-\Adj{} person drink-\Pst{} \Dem{}.\Anim{}.\Dist{}.\Sg{} water \Dem{}.\Anim{}.\Dist{}.\Sg{} \\
        this person drank that water that \\
        \glt
        This person is who drank the water.
    \end{exe}

    A clause acting as a noun (i.e.\ a content clause)
    is expressed by simply placing the subordinate clause directly within the outer clause.
    This is similar to the English form that elides ``that''.

    \begin{exe}
        \ex
        \glt
        xwè wùg krânce gó wùgwùg.
        \glll
        xwè wùg krân-ce gó wùgwùg \\
        know 1\Sg{} see-\Pst{} 3\Sg{}.\Anim{} 1\Pl{} \\
        know me saw him us \\
        \glt
        I know that he saw us.
        /
        I know he saw us.
    \end{exe}

    \subsubsection*{Questions}
    All questions have the question particle \textit{ã} at the front.

    A polar question is formed by adding the question particle \textit{ã}
    at the front of the sentence,
    and adding \textit{ha} ``yes'' or \textit{ìk} ``no'' to the end.
    There is no significant difference between the two options
    (meaning neither is the expected answer).

    \begin{exe}
        \ex
        \glt
        ã, câwtuce gógó wug, ha?
        \glll
        ã câwtu-ce gógó wug ha \\
        \Q{} find-\Pst{} 3\Pl{}.\Anim{} dog yes \\
        \Q{} found them dog yes \\
        \glt
        Did they find the dog?
    \end{exe}

    \begin{exe}
        \ex
        \glt
        ã, kōgce gréryl kǒg?
        \glll
        ã kōg-ce gré-ryl kǒg \\
        \Q{} eat-\Pst{} farm-\Agt{}.\Anim{} grain \\
        \Q{} ate farmer grain \\
        \glt
        Did the farmer eat the grain?
    \end{exe}

    An open question is formed by adding the question particle \textit{ã}
    at the front of the sentence,
    and using the interrogative pronoun \textit{twãx} for inanimate topics
    and \textit{tãx} for animate topics.
    Just like other pronouns, these reduplicate for plurals
    and take the \textit{-hły} adjectival suffix to form determiners.
    (\Int{} means interrogative pronoun.)

    \begin{exe}
        \ex
        \glt
        ã, câwtuce twãx gógóni wug?
        \glll
        ã câwtu-ce twãx gógó=ni wug \\
        \Q{} find-\Pst{} \Int{}.\Inanim{}.\Sg{} 3\Pl{}.\Anim{}=\InessTwo{} dog \\
        \Q{} found what them=at dog \\
        \glt
        Where did they find the dog?
    \end{exe}

    \begin{exe}
        \ex
        \glt
        ã, hłîkce tãxtãx à krângaty sōkro?
        \glll
        ã hłîk-ce tãxtãx à krân-ga=ty sōkro \\
        \Q{} make-\Pst{} \Int{}.\Anim{}.\Pl{} 2\Sg{} see-\Ger{}=\Poss{} leaf \\
        \Q{} made who you seeing=\Poss{} leaf \\
        \glt
        Who (pl.) showed you the leaf?
    \end{exe}

    \begin{exe}
        \ex
        \glt
        ã, kōgce à twãx?
        \glll
        ã kōg-ce à twãx \\
        \Q{} eat-\Pst{} 2\Sg{} \Int{}.\Inanim{}.\Sg{} \\
        \Q{} ate you what \\
        \glt
        What did you eat?
    \end{exe}

    \begin{exe}
        \ex
        \glt
        ã, lāl à twãx-hły lâl?
        \glll
        ã lāl à twãx-hły lâl \\
        \Q{} sing 2\Sg{} \Int{}.\Inanim{}-\Adj{} song \\
        \Q{} sing you what song \\
        \glt
        What song are you singing?
    \end{exe}

    For questions with options,
    simply list the options at the end of the question,
    joining them with ``or''.

    \begin{exe}
        \ex
        ã, kazcǒ à twãxhły, hlātu, cycīl, zyla cec?
        \glll
        ã kazcǒ à twãx-hły hlātu cycīl zyla cec \\
        \Q{} want 2\Sg{} \Int{}.\Inanim{}-\Adj{} plate red blue white or \\
        \Q{} want you what plate red blue white or \\
        \glt
        Which plate do you want, red, blue, or white?
    \end{exe}

    \subsubsection*{Derivational Morphology}

    One instance of derivational morphology
    has already been explained in the previous assignment,
    which is \textit{-hły},
    a suffix that forms adjectives out of materials.
    It is overloaded to create determiner forms
    of demonstrative pronouns.

    \begin{exe}
        \ex
        ã, kazcǒ à twãxhły, hlātu, cycīl, zyla?
        \glll
        ã kazcǒ à twãx-hły hlātu cycīl zyla  \\
        \Q{} want 2\Sg{} \Int{}.\Inanim{}-\Adj{} plate red blue white \\
        build-\Hst{} bird grassy nest, wooden nest, stone nest and \\
        \glt
        The birds built a grass nest, a wooden nest, and a stone nest.
    \end{exe}

    \textit{-kac} takes an adjective and turns it into a verb
    with ``become'', like the intransitive forms of
    the English suffixes ``-ify'' and ``-ize''.
    This is semantically somewhat similar to the auxiliary forms from earlier,
    but I chose a different mechanism because it only applies to adjectives.

    \begin{exe}
        \ex
        cycīlkacce zān.
        \glll
        cycīl-kac-ce zān \\
        red-become-\Pst{} sky \\
        became.red sky \\
        \glt
        The sky became red.
        /
        The sky reddened.
    \end{exe}

    \begin{exe}
        \ex
        kłàzkacce zine wên gūskacce zine.
        \glll
        kłàz-kac-ce zine wên gūs-kac-ce zine \\
        dry-become-\Pst{} 3\Sg{}.\Inanim{} when.then small-become-\Pst{} it \\
        dried it when.then shrank it \\
        \glt
        When it dried, it shrank.
    \end{exe}

    \textit{-ryl} takes a verb and makes an animate agentive noun.
    \textit{-łyr} takes a verb and makes an inanimate agentive noun.

    \begin{exe}
        \ex
        xwè slêxryl kwal.
        \glll
        xwè slêx-ryl kwal \\
        know swim-\Agt{}.\Anim{} water \\
        know swimmer water \\
        \glt
        The swimmer knows the water.
    \end{exe}

    \begin{exe}
        \ex
        sîtce kũx õgce tylǐnłyr kũxhły wùg rõr
        \glll
        sît-ce kũx õg-ce tylǐn-łyr kũx-hły wùg rõr \\
        seek-\Pst{} \Dem{}.\Anim{}.\Sg{} use-\Pst{} dig-\Agt{}.\Inanim{} \Dem{}.\Anim{}-\Adj{} 1\Sg{} gold \\
        sought that used spade that me gold \\
        \glt
        I searched for gold with the spade.
    \end{exe}

    \textit{-zoc} takes a verb or adjective and forms a noun
    representing the process or result of the action (like ``-tion'' or the ``-th'' in ``growth'' and ``theft'')
    or the state of the adjective (like ``-ness'').

    \begin{exe}
        \ex
        kȳhihy kùwix kȳt cỳltyzoc.
        \glll
        kȳhi-hy kùwix kȳt cỳlty-zoc \\
        able-\Neg{} buy 4\Sg{} happy-\Nmlz{} \\
        able-not buy one happiness \\
        \glt
        One cannot buy happiness.
    \end{exe}

    \begin{exe}
        \ex
        ã, twãxtwãx à câwtuzocty?
        \glll
        ã twãxtwãx à câwtu-zoc=ty \\
        \Q{} \Int{}.\Inanim{}.\Pl{} 2\Sg{} find-\Nmlz{}=\Poss{} \\
        \Q{} what you findings=\Poss{} \\
        \glt
        What are your findings?
    \end{exe}

    \textit{-łon} takes a noun and forms an adjective of similarity,
    like ``-like'' in English.

    \begin{exe}
        \ex
        kiłìlłon wùgwùg xwẽ crizǐtyty.
        \glll
        kiłìl-łon wùgwùg xwẽ crizǐ=ty=ty \\
        secret-like 1\Pl{} study house=\Poss{}=\Poss{} \\
        secretive us study house=\Poss{}=\Poss{} \\
        \glt
        Our school is secretive.
    \end{exe}

    \begin{exe}
        \ex
        sû nīłon à wugty.
        \glll
        sû nīłon à wug=ty \\
        very cat-like you dog=\Poss{} \\
        very catlike you dog=\Poss{} \\
        \glt
        Your dog is very catlike.
    \end{exe}

    \textit{-et} takes a verb and forms an adjective that
    usually means a specialized or generalized version of the past participle.
    As in English, the participle applies to the object of a transitive verb
    or the subject of an intransitive verb.
    (Wikipedia calls this behavior some distinction between active and passive uses.)

    \begin{exe}
        \ex
        ōt tùl tùlkacet sōkro kwalty.
        \glll
        ōt tùl tùl-kac-et sōkro kwal=ty \\
        should sweet sweet-become-\Ptcp{} leaf water=\Poss{} \\
        should sweet sweetened leaf water=\Poss{} \\
        \glt
        The sweetened tea should be sweet.
    \end{exe}

    \begin{exe}
        \ex
        ã, kōgwyk à gréet gò, ha?
        \glll
        ã kōg-wyk à gré-et gò ha \\
        \Q{} eat-\Fut{} 2\Sg{} farm-\Ptcp{} fruit yes \\
        \Q{} eat-will you farmed fruit yes \\
        \glt
        Will you eat the farmed fruit?
    \end{exe}

    \begin{exe}
        \ex
        xrôk tāgaet wihǔ.
        \glll
        xrôk tāga-et wihǔ \\
        big grow-\Ptcp{} bird \\
        big grown bird \\
        \glt
        The grown bird is big.
    \end{exe}

    \subsubsection*{Sample Sentences}

    I have intentionally made some words with
    some semantic relation similar but with different tones.
    I explain this as historically old morphology
    that was lost and only reflected in tones,
    then completely obscured after tone changes over time.
    In my conworld, parrots have only recently begun
    speaking human-like languages.
    I explain the very fast rates of change
    as volatility within a very new language system.

    \begin{exe}
        \ex
        kwīn kwĩn.
        \glll
        kwīn kwĩn \\
        sour pickle \\
        sour pickle \\
        \glt
        The pickle is sour.
    \end{exe}

    Since the language does not have zero-valency verbs
    and I do not like English's use of dummy pronouns,
    in cases like weather statements
    I use a reasonable subject.

    \begin{exe}
        \ex
        nùr tyłǎc gás ozkã cìz.
        \glll
        nùr tyłǎc gás ozkã cìz \\
        bright sun but cold air \\
        bright sun but cold air \\
        \glt
        The sun is bright, but it's cold.
    \end{exe}

    Double negation is still negative.
    In this case, it is not required everywhere.
    There are several possibilities that must be clarified.
    \begin{itemize}
        \item ``already-not bloom'' means they are just starting to bloom
        (more idiomatic to use a ``just starting'' adverb).
        \item ``already bloom-not'' means they are no longer blooming.
        \item ``already-not bloom-not'' means have not yet bloomed.
        \item ``already-not bloomed'' means they were starting to bloom.
        \item ``already bloomed-not'' means they had stopped blooming.
        \item ``already-not bloomed-not'' means they had not yet bloomed.
    \end{itemize}
    \begin{exe}
        \ex
        ǐci łŷzce hwàgkin łỹz, gás ǐcihy łŷzhy zyla łỹz.
        \glll
        ǐci łŷz-ce hwàgkin łỹz gás ǐci-hy łŷz-hy zyla łỹz \\
        already bloom-\Pst{} yellow flower but already-\Neg{} bloom-\Neg{} blue flower \\
        already bloomed yellow flower but already-not bloom-not blue flower \\
        \glt
        The yellow flowers have bloomed, but the blue flowers have not.
    \end{exe}

    \begin{exe}
        \ex
        ǐci łŷzhy cycīl łỹz.
        \glll
        ǐci łŷz-hy cycīl łỹz \\
        already bloom-\Neg{} red flower \\
        already bloom-not red flower \\
        \glt
        The red flowers already are not blooming.
        /
        The red flowers have stopped blooming.
    \end{exe}

    To say it is some season, say the day is of the season.
    \begin{exe}
        \ex
        howeg zasõgty kàtty tot cycīlkac sōkro.
        \glll
        howeg zasõg=ty kàt=ty tot cycīl-kac sōkro \\
        wind season=\Poss{} day=\Poss{} and red-become leaf \\
        wind season=\Poss{} day=\Poss{} and redden leaf \\
        \glt
        It is autumn and the leaves are turning red.
    \end{exe}

    \begin{exe}
        \ex
        kōg tyłǎc zasõgty wùgwùgni āx gò.
        \glll
        kōg tyłǎc zasõg=ty wùgwùg=ni āx gò \\
        eat sun season=\Poss{} 1\Pl{}=\InessTwo{} much fruit \\
        eat sun season=\Poss{} us=in much fruit \\
        \glt
        We eat a lot of fruit in summer.
    \end{exe}

    \begin{exe}
        \ex
        xãk gógó kwałŷxty xârtyi.
        \glll
        xãk gógó kwałŷx=ty xâr=ty=i \\
        bottom 3\Pl{}.\Anim{} tree=\Poss{} top=\Poss{}=\IllThree{} \\
        bottom them tree=\Poss{} top=\Poss{}=into \\
        \glt
        Their trees' tops go down.
    \end{exe}

    (The language does have separate words for big, many, and much.)
    \begin{exe}
        \ex
        lỳ sōkro gás łỹz tôctyn sỳcty.
        \glll
        lỳ sōkro gás łỹz tôctyn sỳc=ty \\
        green leaf but flower many color=\Poss{} \\
        green leaf but flower many color=\Poss{} \\
        \glt
        Leaves are green but flowers have many colors.
    \end{exe}

    \begin{exe}
        \ex
        ǐxa łŷz tot slyc słarýhły kwałŷxhły hengó.
        \glll
        ǐxa łŷz tot slyc słarý-hły kwałŷx-hły hengó \\
        sudden bloom and die grass-\Adj{} tree-\Adj{} forest \\
        sudden bloom and die grass tree forest \\
        \glt
        A bamboo forest blooms and dies suddenly.
    \end{exe}

    (
    Wind instruments are called ``[inanimate] singers''
    since, compared to other instruments,
    their way of making sound is similar to that of humans and parrots.
    )
    \begin{exe}
        \ex
        hłīk cìz lālłyr słarýhły kwałŷxhły tàek xrîzgaty.
        \glll
        hłīk cìz lāl-łyr słarý-hły kwałŷx-hły tàek xrîz-ga=ty \\
        make air sing-\Agt{}.\Inanim{} grass-\Adj{} tree-\Adj{} chip shake-\Ger{}=\Poss{} \\
        make air singer grass tree chip shaking=\Poss{} \\
        \glt
        Air shakes a wind instrument's reed.
    \end{exe}

    \begin{exe}
        \ex
        wugłon õgce wùg krũxhły kláte.
        \glll
        wug-łon õg-ce wùg krũx-hły kláte \\
        dog-like use-\Pst{} 1\Sg{} \Dem{}.\Inanim{}.\Dist{}-\Adj{} design \\
        dog-like used me that design \\
        \glt
        The design I used is dog-like.
    \end{exe}

    \begin{exe}
        \ex
        kazcǒhy hàgry kȳtkȳt xār xwẽty.
        \glll
        kazcǒ-hy hàgry kȳtkȳt xār xwẽ=ty \\
        want-not some 4\Pl{} brain study=\Poss{} \\
        want-not some them brain study=\Poss{} \\
        \glt
        Some don't want to use psychology.
    \end{exe}

    \begin{exe}
        \ex
        kazcǒhy hàgry kȳtkȳt xār xwẽty.
        \glll
        xrôł kȳtkȳt kùwix él krârwik \\
        say 4\Pl{} buy person idea \\
        say them buy person idea \\
        \glt
        They say people buy ideas.
    \end{exe}

    \begin{exe}
        \ex
        cỳlty cǎw nùr tot lâl gás kwãlhy zine.
        \glll
        cỳlty cǎw nùr tot lâl gás kwãl-hy zine \\
        happy loud bright and song but good-\Neg{} 3\Sg{}.\Inanim{} \\
        happy loud bright and song but good-not it \\
        \glt
        The song is happy, loud, and bright, but it is not good.
    \end{exe}

    \begin{exe}
        \ex
        cỳlty cǎw nùr tot lâl gás kwãlhy zine.
        \glll
        xwè-ce wùg wán-ce gógó krũx-hły wàn \\
        know-\Pst{} 1\Sg{} play-\Pst{} 3\Pl{}.\Anim{} \Dem{}.\Inanim{}.\Dist{}-\Adj{} game \\
        knew me played them that game \\
        \glt
        I knew the game they were playing.
    \end{exe}

    \begin{exe}
        \ex
        õgce wánryl gūs zynyt.
        \glll
        õg-ce wán-ryl gūs zynyt \\
        use-\Pst{} play-\Agt{}.\Anim{} small stone \\
        used player small stone \\
        \glt
        The players used small stones.
    \end{exe}

    (No, tea is not especially important to parrots.)
    \begin{exe}
        \ex
        hēwak hàgry él lỳ sōkro kwalty, tot hēwak hàgry él słàn sōkro kwalty.
        \glll
        hēwak hàgry él lỳ sōkro kwal=ty tot hēwak hàgry él słàn sōkro kwal=ty \\
        drink some person green leaf water=\Poss{} and drink some person black leaf water=\Poss{} \\
        drink some person green leaf water=\Poss{} and drink some person black leaf water=\Poss{} \\
        \glt
        Some people drink green tea, and some people drink black tea.
    \end{exe}

    \begin{exe}
        \ex
        hēwakce wánce kũxkũx wàn kũxhły él lỳ sōkro kwalty.
        \glll
        hēwak-ce wán-ce kũxkũx wàn kũxhły él lỳ sōkro kwal=ty \\
        drink-\Pst{} play-\Pst{} \Dem{}.\Anim{}.\Dist{}.\Pl{} game \Dem{}.\Anim{}.\Dist{}-\Adj{} person green leaf water=\Poss{} \\
        drank played that game those people green leaf water=\Poss{} \\
        \glt
        The people playing the game were drinking green tea.
    \end{exe}

    \begin{exe}
        \ex
        hēwakce wánce kũxkũx wàn kũxhły él lỳ sōkro kwalty.
        \glll
        hēwak-ce wán-ce kũxkũx wàn kũxhły él lỳ sōkro kwal=ty \\
        drink-\Pst{} play-\Pst{} \Dem{}.\Anim{}.\Dist{}.\Pl{} game \Dem{}.\Anim{}.\Dist{}-\Adj{} person green leaf water=\Poss{} \\
        drank played that game those people green leaf water=\Poss{} \\
        \glt
        The people playing the game were drinking green tea.
    \end{exe}

    \begin{exe}
        \ex
        tāga rỳt lúk sōkro kwalty kwałŷxtyni.
        \glll
        tāga rỳt lúk sōkro kwal=ty kwałŷx=ty=ni \\
        grow warm place leaf water=\Poss{} tree=\Poss{}=\InessTwo{} \\
        grow warm place leaf water=\Poss{} tree=\Poss{}=in \\
        \glt
        The tea tree grows in warm places.
        [geographically]
    \end{exe}

    \begin{exe}
        \ex
        tīke slôrslôr rõk ozkāla.
        \glll
        tīke slôrslôr rõk ozkā=la \\
        white \Dem{}.\Inanim{}.\Prox{}.\Pl{} mountain ice=\AdessThree{} \\
        white these mountains ice=on \\
        \glt
        The ice on these mountains is white.
    \end{exe}

    \begin{exe}
        \ex
        kȳhi sitǎr zèkhły xār tôctyn krârwik kȳt.
        \glll
        kȳhi sitǎr zèk-hły xār tôctyn krârwik kȳt \\
        able show electricity-\Adj{} brain many idea 4\Sg{} \\
        able show electric brain many idea one \\
        \glt
        The computer can show one many ideas.
    \end{exe}

    \begin{exe}
        \ex
        hłīkwyk sotê hūg slycgaty.
        \glll
        hłīk-wyk sotê hūg slyc-ga=ty \\
        make-\Fut{} rain fire die-\Ger{}=\Poss{} \\
        make.will rain fire dying=\Poss{} \\
        \glt
        The rain will put out the fire.
        /
        The rain will kill the fire.
    \end{exe}

    \begin{exe}
        \ex
        câwtuce gó kwałŷx xãkty xełla.
        \glll
        câwtu-ce gó kwałŷx xãk=ty xeł=la \\
        find-\Pst{} 3\Sg{}.\Anim{} tree bottom=\Poss{} \Refl{}.\Sg{}=\AdessThree{} \\
        found him tree bottom=\Poss{} self=at \\
        \glt
        He found himself under the tree.
    \end{exe}

    \begin{exe}
        \ex
        kwǒnce kwałŷx xãkty góla.
        \glll
        kwǒn-ce kwałŷx xãk=ty gó=la \\
        sleep-\Pst{} tree bottom=\Poss{} 3\Sg{}.\Anim{}=\AdessThree{} \\
        slept tree bottom=\Poss{} him=at \\
        \glt
        He slept under the tree.
    \end{exe}

    \begin{exe}
        \ex
        kwałŷx xãkty crákli.
        \glll
        kwałŷx xãk=ty crák=li \\
        tree bottom=\Poss{} ant=\InessThree{} \\
        tree bottom=\Poss{} ant=in \\
        \glt
        The ant was in the bottom part of the tree.
    \end{exe}

    \begin{exe}
        \ex
        sît hàgry crák sōkro gás kȳhi kōg gógó crò ǐ \\
        \glll
        sît hàgry crák sōkro gás kȳhi kōg gógó crò ǐ \\
        seek some ant leaf but able eat 3\Pl{}.\Anim{} other thing \\
        seek some ant leaf but able eat them other things \\
        \glt
        Some ants search for leaves but they can eat other things.
    \end{exe}

    \begin{exe}
        \ex
        cohon crò crák tùl kwal
        \glll
        cohon crò crák tùl kwal \\
        create other ant sweet water \\
        create other ant sweet water. \\
        \glt
        Other ants make sweet water.
    \end{exe}

    \begin{exe}
        \ex
        xrôk ìl.
        \glll
        xrôk ìl \\
        big language \\
        big language \\
        \glt
        Languages are big.
    \end{exe}

    \begin{exe}
        \ex
        xwèce zàce kũx xrôłga kũxhły wihǔ kiłìl.
        \glll
        xwè-ce zà-ce kũx xrôł-ga kũx-hły wihǔ kiłìl \\
        know-\Pst{} do-\Pst{} \Dem{}.\Anim{}.\Dist{}.\Sg{} say-\Ger{} \Dem{}.\Anim{}.\Dist{}-\Adj{} bird secret \\
        knew did that saying that bird secret \\
        \glt
        The bird who spoke knew a secret.
    \end{exe}

    \begin{exe}
        \ex
        gāwce gó zine wihǔ gás gāwcehy gó zine hełǎ.
        \glll
        gāw-ce gó zine wihǔ gás gāw-ce-hy gó zine hełǎ \\
        give-\Pst{} 3\Sg{}.\Anim{} 3\Sg{}.\Inanim{} bird but give-\Pst{}-\Neg{} 3\Sg{}.\Anim{} 3\Sg{}.\Inanim{} human \\
        gave him bird but gave-not him human \\
        \glt
        He gave it to the birds but not the humans.
    \end{exe}
\end{document}