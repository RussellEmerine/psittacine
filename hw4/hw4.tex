\documentclass[12pt]{article}
\pagestyle{plain}
\usepackage[letterpaper,margin=1in]{geometry}
\usepackage{tipa}
\usepackage{amsmath}
\usepackage{leipzig}
\usepackage{gb4e}

\newleipzig{InessTwo}{iness2}{2d inessive}
\newleipzig{InessThree}{iness3}{3d inessive}
\newleipzig{AdessTwo}{adess2}{2d adessive}
\newleipzig{AdessThree}{adess3}{3d adessive}
\newleipzig{IllThree}{ill3}{3d illative}
\newleipzig{AllTwo}{all2}{2d allative}
\newleipzig{AllThree}{all3}{3d allative}
\newleipzig{AblThree}{abl3}{3d ablative}

\makeglossaries

% Document
\begin{document}

    While browsing Wikipedia at one point,
    I learned that some languages have a ``construct state''
    which involves modifying a noun to indicate
    that it is possessed by another noun.
    In Arabic, this process is called \textit{iḍāfah}.
    One example Wikipedia gives in Egyptian Arabic is
    \begin{center}
        \begin{tabular}{|c|c|}
            \hline
            malika    & a queen              \\
            \hline
            il-malika & the queen            \\
            \hline
            malik(i)t & a/the queen of \dots \\
            \hline
        \end{tabular}
    \end{center}
    ((i) is present or absent according to sandhi.)

    I found this translation with ``of \dots'' unusual
    and decided it would be interesting to generalize this to
    all positional relationships.
    When relating some noun (or noun phrase) to some location (or possessor),
    rather than keeping the noun the same
    and attaching an adposition to the location,
    I will modify the noun and keep the location the same.
    As far as I'm aware, no natural language does this,
    so I get to invent terminology!
    I'll call the modification of the noun
    a ``stance form'',
    since it indicated how the noun is positioned,
    which is kind of like a stance.

    I also decided that since birds can fly,
    they live more three-dimensional lives than humans.
    Then, it is plausible that they would have
    more positional relations than humans,
    including a distinction between three-dimensional ``in'' and ``on''
    and two-dimensional ``in'' and ``on''.
    For example,
    there may be walnuts three-dimensionally in a loaf of banana bread,
    while Rome is two-dimensionally in Italy (as on a map).

    \textit{
        (The inspiration for this is that in middle school,
        I got a 3D chess variant called YAVOCH.
        The lore in YAVOCH is that aliens are confronting humans
        and have given humans spaceships to allow a fair fight.
        Humans were unable to pilot the spaceships until
        they connected animal brain patterns to the systems.
        In particular,
        the animal brains (including birds) were better with 3D spatial movement
        than human logic was.)
    }

    \textit{
        (There was also a time in a math class long ago when the teacher
        referenced a point ``on'' a circle,
        meaning on the outline,
        but many students assumed it was an interior point.
        I thought it would be interesting to have a distinction.)
    }

    Stance forms will cover the following semantic roles,
    with corresponding English forms:
    \begin{itemize}
        \item
        Location of being
        \begin{itemize}
            \item 2D in, 2D on, 3D in, 3D on
            \item over, under, near, far from
        \end{itemize}
        \item
        Location of motion, as origin, destination, or path
        \begin{itemize}
            \item 2D into, onto, out of, off of, through, across, and 3D \dots
            \item towards, away from
        \end{itemize}
        \item
        Possession
    \end{itemize}

    Stance forms can also be used with partial nouns,
    like ``top'' or ``side'',
    which each are connected to their own nouns as possessed forms.
    The same partial noun can take on different meanings according to 2D or 3D interpretation.
    This can lead to very detailed descriptions of position or movement,
    such as going up a mountain to reach its top face (2D ``top''),
    vs.\ going up a mountain to fly in the space at its peak (3D ``top'').

    I have not decided on verb morphology yet,
    but using a stance form without a verb will translate to
    locational ``be'',
    ``go'', or possession,
    according to the meaning.
    This can be thought of as a null copula.

    The specific morphology of stance forms will be a simple suffix clitic.
    The syntax is that the location (or possessor) will come first,
    and the noun with the stance form will come second.
    Also, I have decided on (for now) no verb inflection for person and number,
    no determiners,
    and no noun inflection for case or number.

    I have chosen the following sentences for my sample:

    \begin{exe}
        \ex
        \glll
        hengó wùg=ni \\
        forest \First{}\Sg{}=\InessTwo{}\\
        forest I=in \\
        \glt
        I am in the forest.
    \end{exe}

    Large areas like forests and countries are considered two-dimensional
    since they cover a flat area of the earth.

    \begin{exe}
        \ex
        \glll
        rõk kwałŷx=la wùg lelō=ty=li \\
        mountain tree=\AdessThree{} \First{}\Sg{} nest=\Poss{}=\InessThree{} \\
        mountain tree=on I nest=\Poss{}=in \\
        \glt
        My nest is in the tree on the mountain.
    \end{exe}

    Trees and mountains are considered three-dimensional.
    Nestled stance phrases behave as expected.
    Here, ``nest'' is part of two stance phrases,
    one of possession and one of location.
    The stance suffixes are placed one after the other,
    and the corresponding nouns are determined by order and context.

    \begin{exe}
        \ex
        \glll
        kwałŷx xâr=ty słygiz=la \\
        tree top=\Poss{} vine=\AdessThree{} \\
        tree top=\Poss{} vine=towards \\
        \glt
        The vine goes up the tree.
        \ex
        \glll
        crizǐ xâr=ty słygiz=na \\
        house top=\Poss{} vine=\AdessTwo{} \\
        house top=\Poss{} vine=towards \\
        \glt
        The vine goes up the house.
    \end{exe}

    When a vine goes up a tree,
    it grows outwards in the space near the top of the tree,
    spreading in all directions.
    When a vine goes up a house,
    it lays flat on the roof,
    restricted to the plane the roof is in.
    So, ``top'' is used as 3D for the example with the tree,
    and as 2D for the example with the house.

    \begin{exe}
        \ex
        \glll
        hlōs nūr=za \\
        screen light=\AllTwo{} \\
        screen light=onto \\
        \glt
        The light [of a flashlight] goes onto the screen.
        \ex
        \glll
        crizǐ nūr=a \\
        house light=\AllThree{} \\
        house light=onto \\
        \glt
        The light [of a flashlight] goes onto the house.
        \ex
        \glll
        kłǎs nūr=i \\
        glass light=\IllThree{} \\
        glass light=into \\
        \glt
        The light [of a flashlight] goes into the glass.
    \end{exe}

    Light projecting onto objects demonstrates the dimensional distinction well.
    Screens are two-dimensional, and light goes to the interior of the surface area (rather than the edge).
    Boxes are three-dimensional, and light goes to the exterior of the object.
    Glass is three-dimensional, and light goes to the interior of the object.

    \begin{exe}
        \ex
        \glll
        rõk howeg=nosa \\
        mountain wind=\AblThree \\
        mountain wind=from \\
        \glt
        Wind comes from the mountain.
    \end{exe}

    This particular sentence could be interpreted literally,
    or idiomatically as ``Mountain-climbing is difficult'',
    or perhaps as something else compared to mountain-climbing is difficult.
    The use of a 3D affix almost suggests that the wind comes from within the mountain,
    which is semantically interesting,
    though it is also reasonable that the form just happens to be used that way.

    ~

    All nonstandard Leipzig-style abbreviations used in the examples are listed.
    There are more unlisted cases necessary, but nothing surprising,
    just the expected lists for standard ablative, adessive, allative, elative, illative, and inessive forms
    (and also maybe path forms).
    These are \textit{not} noun cases, as they don't attach to locations.
    I am overloading the notation to express similar spatial relations.

    \printglossaries

\end{document}