\documentclass[12pt]{article}

\usepackage[letterpaper,margin=1in]{geometry}
\usepackage{array}
\usepackage{amsmath}

% Document
\begin{document}
    Cognitive metaphor: \textsc{Experiences are flights}

    Parrots are flying creatures.
    While they are adept climbers without their wings,
    they rely on flight for almost all their movement,
    as much as humans rely on walking.
    This leads to pervasive metaphor involving flight.

    \begin{itemize}
        \item
        \textit{There's a headwind/tailwind northward.}

        \dots or in whichever direction movement is desired,
        perhaps another cardinal direction
        (which would be understood to be alluding to the idiom),
        or something more specific to the situation,
        perhaps ``profitward'' if you're building a business,
        or ``skillward'' if you're training to learn something.
        The feel I'm aiming for could possibly be achieved with an allative case
        placed on things that normally don't take an allative case.

        The meaning of this metaphor is that the future experience will be
        hard and long/easy and short.
        This is like the English metaphor of ``the path ahead is long and difficult'',
        but parrots don't use paths, they fly.
        A headwind will make flight slower and more difficult,
        while a tailwind will make flight faster and easier.

        \item
        \textit{to fly with one wing}

        The meaning of this metaphor is to not put in effort.
        Humans can make some attempt to run with one leg,
        but parrots absolutely cannot fly with one wing.
        So, if you fly with one wing,
        you won't accomplish anything.
        Interestingly, I couldn't find any equivalent metaphors
        related to walking in any languages I know.
        It does bear some resemblance to ``half-baked'' or ``half-assed'' though.

        (Fun fact: according to Wiktionary,
        ``half-assed'' is etymologically a humorous alternation of ``haphazard''.)

        \item
        Additionally there will be some double meanings in common vocabulary based on experiences as flights.

        \begin{center}
            \begin{tabular}{b{5em} b{27em}}
                \hline
                glide        & have an easy job, like ``cruise'' in English                                                    \\
                \hline
                take off     & get started, gain momentum, like ``take off'' in English, which is also a flight-based metaphor \\
                \hline
                land, alight & finish                                                                                          \\
                \hline
                fly up       & search, or gain perspective                                                                     \\
                \hline
            \end{tabular}
        \end{center}

    \end{itemize}

    \newpage
    Cognitive metaphor: \textsc{Speech is song}

    While researching, I learned that parrots' logical thinking does not occur in the cerebral cortex,
    but rather in the HVC (an acronym that no longer has meaning),
    which is the area of the brain responsible for birdsong.
    Parrots famously can copy human speech,
    but they can also copy musical sounds and melodies.
    So, the parrots in my world will have no significant distinction between language, birdsong, and music.

    \begin{itemize}
        \item
        \textit{to have/lack a melody}

        The meaning of this metaphor is to have/lack a point in an story, discussion, or argument.
        In English, we might instead use ``theme'', ``motif'', or ``point''.
        The first two happen to also be used for music,
        but they are general terms that apply to artwork, ideas, science, and so on.
        In my language, ``melody'' will only apply to music and speech.

        \item
        \textit{to sing someone's song}

        The meaning of this metaphor is to advocate for someone else's ideas.
        It's also possible \textit{to sing a different song [from other birds]},
        which means to have a novel, radical idea.
        Both forms can be with a positive or negative connotation.

        \item
        Additionally there will be some double meanings in common vocabulary based on experiences as flights.

        \begin{center}
            \begin{tabular}{b{5em} b{27em}}
                \hline
                language \textit{or} birdsong \textit{or} music
                & these will all be the same word, distinguished by adjectives when necessary \\
                \hline
                twitter
                & talk annoyingly, like ``yap'' in English, which is based on dogs            \\
                \hline
                rhythm
                & consistency, reliability                                                    \\
                \hline
                writing
                & ``etched song'', since they write by scratching their beaks on bark         \\
                \hline
                musical \quad instrument
                & some form of ``talker''                                                     \\
                \hline
            \end{tabular}
        \end{center}
    \end{itemize}

\end{document}